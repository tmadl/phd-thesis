\documentclass[12pt,PhD,twoside]{muthesis}
% The regulations say that 12pt should be used
% Change the MSc option to MPhil, MRes or PhD if appropriate

\usepackage{verbatim}
\usepackage{graphicx}
\usepackage{url} % typeset URL's reasonably
\usepackage{listings}

\usepackage{pdfpages}

\usepackage{tabu}
\usepackage{longtable}
\usepackage{multirow}
\usepackage[labelfont=bf]{caption}

\usepackage{natbib}

\usepackage{pslatex} % Use Postscript fonts

% amsmath package, useful for mathematical formulas
\usepackage{amsmath}
% amssymb package, useful for mathematical symbols
\usepackage{amssymb}

\usepackage{bm}

\usepackage{pseudocode}

\DeclareMathOperator*{\argmin}{arg\,min}
\DeclareMathOperator*{\argmax}{arg\,max}

\def\approxprop{%
	\def\p{%
		\setbox0=\vbox{\hbox{$\propto$}}%
		\ht0=0.6ex \box0 }%
	\def\s{%
		\vbox{\hbox{$\sim$}}%
	}%
	\mathrel{\raisebox{0.7ex}{%
			\mbox{$\underset{\s}{\p}$}%
		}}%
	}

\usepackage{tocloft}
\setlength{\cfttabindent}{0in}
\setlength{\cftfigindent}{0in}

% Uncomment the next line if you want subsubsections to be numbered
%\setcounter{secnumdepth}{3}
% Uncomment the next line if you want subsubsections to be appear in
% the table of contents
%\setcounter{tocdepth}{3}

% Uncomment the following lines if you want to include the date as a
% header in draft versions
%\usepackage{fancyhdr}
%\pagestyle{fancy}
%\lhead{}  % left head
%\chead{Draft: \today} % centre head
%\lfoot{}
%\cfoot{\thepage}
%\rfoot{}

\begin{document}
% Uncomment the following lines to leave out list of figures, tables
% and copyright until final printing
%\figurespagefalse
%\tablespagefalse
%\copyrightfalse

\title{Bayesian mechanisms in spatial cognition:\\
Towards real-world capable computational cognitive models of spatial memory}
\author{Tamas Madl}
%\principaladviser{Ke Chen}

\beforeabstract

\prefacesection{Abstract}
\abstracttitle


%Abstract
%
%A good abstract explains in one line why the paper is important. It then goes on to give a summary of your major results, preferably couched in numbers with error limits. The final sentences explain the major implications of your work. A good abstract is concise, readable, and quantitative. 
%Length should be ~ 1-2 paragraphs, approx. 400 words. 
%Absrtracts generally do not have citations.
%Information in title should not be repeated. 
%Be explicit. 
%Use numbers where appropriate.
%Answers to these questions should be found in the abstract: 
%What did you do? 
%Why did you do it? What question were you trying to answer? 
%How did you do it? State methods.
%What did you learn? State major results. 
%Why does it matter? Point out at least one significant implication.

% Single spacing can be turned on for the abstract
%
{
\singlespacing
\setstretch{1.0}
Computational cognitive models of spatial memory often neglect difficulties posed by the real world, such as sensory noise, uncertainty, and high spatial complexity. However, since cognition and its neural bases have been shaped by the structure and challenges of the physical world, cognitive models should take them into account.

This work takes an interdisciplinary approach towards developing a cognitively plausible spatial memory model able to function in realistic environments, despite sensory noise and spatial complexity. We investigated how spatially relevant brain areas might maintain accurate location estimates, despite accumulating sensory noise, hypothesizing that hippocampal place cells might perform Bayesian localization, and that hippocampal reverse replay might play a role in correcting learned maps after revisiting known places. We developed computational models implementing these probabilistic mechanisms, which we argued to be psychologically plausible (producing human-like behaviour) as well as neurally plausible (implementable in brains). In support of the hippocampal Bayesian localization hypothesis, we reported modelling results of single-neuron recordings from rats (acquired outside this PhD), constituting the first evidence for Bayesian inference in place cells, as well as modelling behaviour data from humans. We also collected and modelled sketch map accuracy data in experiments performed online, substantiating the suggested map correction mechanism.

In addition to dealing with noise and uncertainty, in realistic environments, large-scale representations also have to be stored and used efficiently. Hierarchical representations help dealing with large amounts of spatial information by facilitating rapid and efficient retrieval search and route planning. It has been suggested that cognitive maps in humans are hierarchical, but the computational principles underlying these hierarchies have remained unknown. We investigated features influencing cognitive map structure using spatial memory data concerning real-world and virtual reality environments collected in online experiments, and proposed a computational mechanism (clustering in psychological space) which might give rise to sub-map structures, showing that it can predict these structures in participants' spatial memory in advance.
%We validated our proposed mechanism using spatial memories of human subjects in over a hundred cities world-wide, and implemented a computational model able to predict, in advance, their sub-map structures based on our hypothesis.

%\vspace{0.8em}

We have extended a general cognitive architecture (the LIDA model of cognition) by these Bayesian mechanisms for localization and map learning, correction, and structuring; integrating them with the other cognitive phenomena accounted for by LIDA. We demonstrated the ability of the resulting model to deal with the challenges of realistic environments by running it in high-fidelity robotic simulations, modelled after participants' actual cities, showing that it can deal with noise, uncertainty and complexity, and that it can reproduce the spatial accuracies of human participants. 
%Our LIDA-based cognitive software agent could reproduce the spatial errors of human participants in different recreated environments, substantiating the plausibility of the suggested models.

}


\afterabstract

\prefacesection{Acknowledgements}
I would like to thank...
\afterpreface

% These include the actual text
\chapter{Introduction}
\label{cha:intro}

%\section{Spatial cognition and probability}
%\label{sec:intro:brainspaceprob}

Brains have evolved to move bodies through space in order to increase the chances of survival and reproduction, through numerous complex behaviours such as fleeing from threats or searching for nutrients or potential mates. The ability to remember spatial information, e.g. previously encountered food sources or shelters, has provided sufficient evolutionary advantage that all known organisms with brains (and even some without, such as the slime mold\footnote{Slime molds are able to avoid previously explored areas using externalized spatial memories, and to solve mazes using nutrient gradients} - \citet{reid2012slime}) have at least a rudimentary ability to utilize representations of space for more efficient navigation. Higher mammals have evolved a network of brain areas implementing spatial memory, a system for storing and recalling spatial information about the environment and about their location in it.

Representing spatial information accurately in the real world is hard, for several reasons. Sensors and actuators are limited, erroneous and noisy (in the sense of noise interfering with the signal). There are additional sources of uncertainty or unknown information, such as external events, actions of other organisms, unperceived or currently unperceivable objects or events. Furthermore, physical environments can be highly complex, and yet cognitive resources (amount of memory, processing power, time and energy available) are necessarily limited by biological and physical constraints. 

In artificial intelligence (AI) and robotics research, probabilistic models have provided key tools for dealing with such challenges, facilitating the quantitative characterization of beliefs and uncertainty in the form of probability distributions, and the machinery of Bayesian inference for updating them with new data. They have also inspired the `Bayesian brain' \citep{knill2004bayesian} and `Bayesian cognition' \citep{chater2010bayesian} paradigms in the cognitive sciences. These paradigms have been successful in explaining human behaviour in tasks as diverse as the integration of sensory cues \citep{ernst2006bayesian} including spatial information \citep{cheng2007bayesian,nardini2008development}, sensorimotor learning \citep{kording2004bayesian}, visual perception \citep{yuille2006vision} or reasoning \citep{oaksford2007bayesian}. Their success suggests an answer to what biological cognition might be doing to cope with the above-mentioned challenges: approximate Bayesian inference.

\begin{figure}[h]
	\centering
	\includegraphics[width=\textwidth]{img/motivation2}
	\caption[Motivation for proposing new computational cognitive models of spatial memory]{\textbf{Motivation for proposing new computational cognitive models of spatial memory}. A: Learning representations of the space around animals confers significant advantages, such as the ability to plan a detour out of sight (dashed red arrow) to reach a food source while avoiding danger in this example. In real environments, this task is made more difficult by the unreliability, errors and noise inherent in both the estimation of position by integrating self-motion and in estimated object distances (e.g. based on vision). Most existing cognitive models of spatial memory neglect these challenges. B: State of the art SLAM models in robotics are able to estimate locations and learn maps accurately, but rely on sensors and computations which are very different from biology - e.g. higher measurement accuracy using laser-based distance sensors (LIDAR), centralized control and coordination, and high number of serial operations per second - up to $10^{10}$ floating-point operations per second (FLOPS) needed for state of the art SLAM systems \citep{machado2013evaluation}. C: In contrast, the hippocampus - the major brain area involved in world-centered spatial representations - contains only a few million neurons, of which only a subset is active at a time, each firing only a few times per second \citep{rapp1996preserved,vsimic1997volume}; and relies on noisy, inaccurate sensory measurements. Although many models of spatial memory in brains exist, there is a lack of computational mechanisms which are both neurally and psychologically plausible, and can work in realistic environments and with noisy sensors. (Example SLAM data in Panel B from \citep{newman2011describing}, and 3D rat brain in Panel C from \citep{calabrese2013ontology}, with permission.)} 
	\label{fig:motivation}
\end{figure}

\section{Motivation}
\label{sec:intro:motivation}

Despite of this success and of the suitability of probabilistic models to deal with uncertain and noisy spatial information, there have been few attempts to use them for modelling spatial memory within cognitive modelling, the branch of cognitive science concerned with computationally simulating mental processes. There is a gap in the literature between probabilistic spatial models in robotics and computational cognitive models of spatial memory. In robotics, Simultaneous Localization and Mapping (SLAM) models \citep{thrun2008simultaneous} are capable of dealing with real-world noise, uncertainty, and complexity to some extent, but are cognitively implausible\footnote{In our usage of the terms, a computational model is `psychologically plausible' (or `cognitively plausible') to the extent that it is consistent with psychological findings and can accurately reproduce psychology data, i.e. behaviours. Analogously, it is `biologically plausible' (or `neurally plausible') to the extent that it is consistent with neuroscience and can reproduce neural data, e.g. single-cell recordings or brain imaging results.}. On the other hand,  current computational cognitive models of spatial memory, which are designed to model biological spatial cognition,  cannot deal with all of these challenges, and are thus confined to simplistic simulations (see Chapter \ref{cha:nnreview} for a review, and Figre \ref{fig:motivation} for an overview of the importance of spatial memory and the differences between information processing in robots and brains). 

%There is also a dearth of models of \textit{how} neurons could be able to perform Bayesian inference to improve spatial representations, and of evidence on a neuronal level (as opposed to behavioural) that they can. 

In addition, although spatial representations in humans have been argued early to be hierarchical \citep{hirtle1985evidence, mcnamara1989subjective, greenauer2010micro}, similarly to some robotic implementations having to deal with large, complex environments \citep{kuipers2000spatial, wurm2010octomap}, it is not known how (by which process) these hierarchical spatial maps might be structured. Although many computational models of spatial memory running in simplified environments exist, there is a lack of biologically and psychologically plausible `algorithms' serving as models of human cognitive computations related to spatial information processing which can 

\clearpage

\noindent function in realistic, uncertain, complex environments.

The deprioritization of the problems of uncertainty and noise in favour of tractably modelling other human cognitive mechanisms is also pronounced in cognitive architectures, which try to account for a large number of mental processes in a unified, comprehensive, systems-level model (as opposed to computational cognitive models, which usually focus on a single phenomenon). In their overview of the field, \cite{langley2009cognitive} argue that \textit{`` we should attempt to unify many findings into a single theoretical framework, then proceed to test and refine that theory''}, supporting the arguments of \cite{newell1973you} that \textit{``you can't play 20 questions with nature and win''}, highlighting the importance of systems-level research in the cognitive sciences. Although a few such cognitive architectures do model spatial mechanisms in navigation space \citep{harrison2003act,schultheis2011casimir,sun2004top}, they all run in simple, noise-free environments. According to a comparative table of cognitive architectures \citep{samsonovich2011comparative} available in updated form online\footnote{http://bicasociety.org/cogarch/architectures.htm}, there is currently no cognitive architecture implementing both Bayesian update and an empirically validated, psychologically plausible `cognitive map' at the same time.
%\footnote{CogPrime \citep{goertzel2013cogprime} claims to implement both Bayesian update and cognitive maps, but neither of these mechanisms have been evaluated against human data, or indeed claim to be modelling human cognitive phenomena at all. Instead, CogPrime aims for artificial general intelligence, as opposed to closely adhering to human cognition.}.

The present work was motivated by these gaps in the literature, and aims to take computational cognitive models of navigation-scale\footnote{Human cognition needs to keep track of the space of navigation as well as the spaces immediately around the body (e.g. reachable objects) and of the body (e.g. body-part configurations). Although uncertainty and noise play are important in the latter two spaces as well, we will confine ourselves to navigation-scale spatial mechanisms in this work.} spatial memory one step closer to modelling behaviour in realistic environments, such as high-fidelity robotic simulations or physical environments. It aims to do so by means of proposing probabilistic mechanisms of spatial cognition which are implementable in brains and can reproduce behaviour data, and by computationally implementing these mechanisms, in the form of cognitive models and within an existing cognitive architecture. Situated within the computational sub-fields of cognitive science (cognitive modelling and cognitive architectures), the goal of this work is to contribute to the understanding of information processing in human cognition. As such, although it is computational in nature, the extent of its success is determined by its ability to predict and explain the kinds of behaviour data it is intended to model, as well as its consistency with established findings in psychology and neuroscience. It is not aiming to maximize the accuracy of learned spatial representations, unlike robotics. Neither does it aim for neurobiological fidelity at the cellular level or below. Although building on neuroscientific evidence, our concern is modelling spatial information processing on Marr's algorithmic level of analysis \citep{marr1976understanding, marr1977understanding}, as opposed to e.g. biological neural networks - see Table \ref{tbl:marr} -, with Chapter \ref{cha:bayespc} being the single exception. 

\begin{table*}[h]
	\centering
	{\renewcommand{\arraystretch}{1.2}
		\begin{tabu}{c|c|c}
			$\downarrow$ {Level of analysis} & {Description} & {In this work}\\ \tabucline[3pt]{-}
			1. Computational & \begin{tabular}[c]{@{}c@{}} What problem(s) does the \\ system solve, and why? \end{tabular} & \begin{tabular}[c]{@{}c@{}} Localization,\\ Map error correction, \\ Map structuring \end{tabular} \\\hline
			\begin{tabular}[c]{@{}c@{}} \textbf{2. Algorithmic/} \\ \textbf{Representational} \end{tabular} & \begin{tabular}[c]{@{}c@{}} How might it solve them? (Using\\ what representations and processes?) \end{tabular} & \begin{tabular}[c]{@{}c@{}} Cognitive models \\ of spatial memory \end{tabular} \\\hline
			3. Implementation & How is it implemented physically? & \begin{tabular}[c]{@{}c@{}} Place, grid,  head- \\ direction, border cells, \\ ... \citep{hartley2014space} \end{tabular} \\
		\end{tabu}
	}
	\caption[Investigating spatial mechanisms on Marr's (1976) levels of analysis]{\textbf{Investigating spatial mechanisms on Marr's (1976) levels of analysis}. The present work is mostly concerned with the second level.}
	\label{tbl:marr}
\end{table*}
%\begin{tabular}[c]{@{}c@{}} Approximately Bayes- \\optimal integration\\of information \end{tabular}

% Although this mechanism has been empirically substantiated on a behavioural level before \citep{cheng2007bayesian, nardini2008development}, its

We have investigated the plausibility of Bayesian spatial cue integration both on Marr's algorithmic (Chapter \ref{cha:lida}) and implementation level (Chapter \ref{cha:bayespc}), in order to maintain the desirable criteria of both psychological and neural plausibility for our other models. The possible neural implementation of this mechanism has been unknown, with current mechanistic models of Bayesian inference in brains making assumptions not fully consistent with the anatomy or activity of the hippocampal complex (the major brain areas representing world-centered spatial information) - see next Section. This doubt of biological implementability has motivated our investigation of single-cell electrophysiological data (acquired outside this PhD) to provide the first evidence for Bayesian updating in the hippocampus on a neuronal level, and our proposal of a plausible mechanism for implementing it. This evidence, presented in Chapter \ref{cha:bayespc}, affords a degree of biological plausibility to the models utilizing Bayesian mechanisms in the rest of our work (which is concerned with processes on the algorithmic/representational level).

%especially if behavioural evidence is inconclusive or insufficient to constrain the space of possible models to a concrete implementation

\section{Probabilistic models of space in brains and minds}
\label{sec:intro:uncertaintybrain} 

Although the focus of most of this work is on the computational modelling of behaviour data, we would like the employed mechanisms to be plausibly implementable in the parts of the brain they functionally correspond to. Apart from the lack of neuronal-level evidence that the hippocampal complex may perform Bayesian inference or even represent uncertainty, the possibility of the implementation of such a mechanism given the anatomical and electrophysiological constraints of this network of brain cells is also unclear. 

Below, we briefly review probabilistic neural spatial models which have been proposed in the literature (Chapter \ref{cha:nnreview} provides a more general review of computational cognitive models of spatial memory). We start with normative models of dealing with spatial uncertainty, which derive optimal solutions to the problem a system might be solving (Marr's computational level). We then continue describing mechanistic (implementation level) models which might facilitate these, and their consistency with what is known about the hippocampal complex. More extensive reviews of Bayesian models in brains can be found in \citep{pouget2013probabilistic, vilares2011bayesian}. There is currently little experimental support for any of the proposed neural uncertainty representations. % \citep{pouget2013probabilistic}. 

%In addition to a large number of non-probabilistic computational cognitive models focusing on accounting for specific mechanisms of spatial cognition (see Chapter \ref{cha:nnreview} for a review), a few authors have suggested probabilistic mechanisms the brain might employ which can be used to model spatial cognition.

Models of probabilistic estimation of spatial information have been pioneered by \citep{bousquet1997hippocampus}, who suggested to use a Kalman filter to model localization in the hippocampus. A Kalman filter is a dynamic Bayesian inference algorithm for estimating the values of unknown, not directly observable variables (such as location) from noisy observations, yielding statistically optimal estimates if the noise is normally distributed \citep{kalman1960new}. \citet{macneilage2008computational} also put forth arguments for dynamic Bayesian inference as a model of spatial orientation. They mention both Kalman filters and particle filtering (a related Bayesian filtering algorithm using samples instead of parameters to represent probability distributions), but leave the question of their neural implementation open. Particle filter-based models of localization on the algorithmic level have been suggested by \citep{fox2010hippocampus, cheung2012maintaining}. \citet{osborn2010kalman} went beyond localization, suggesting a Kalman filtering approach to also account for localizing objects in the environment. Recently, \citet{penny2013forward} argued that if one presupposes the existence of `observation' and `dynamic' models\footnote{Observation models and dynamic models are mathematical functions mapping from true states to observed states, and from pre-motion to post-motion states, respectively.}, required by Kalman filters, one might as well extend the inference to also use them for model selection (`which environment am I in?'), motor planning (`how do I get to place X?'), and to construct sensory imagery (`what does place X look like?') in addition to localization. They have combined these functions in a single probabilistic model, and argued that it is consistent with findings of pattern replay in the brain. An even more general probabilistic formulation based on dynamic Bayesian inference is the Free-Energy Principle \citep{friston2006free}, which aspires to provide a unified theory of brain function, and has been argued to be consistent with aspects of hippocampal processing \citep{friston2011action}.

Despite their considerable theoretical elegance, the above-mentioned models do not provide a final and complete answer to the motivating question of this thesis (Section \ref{sec:intro:motivation}), which can be summarized as: `how does biological cognition learn representations of navigation space from noisy sensors in an uncertain world?', for two reasons. First, none of them try to reproduce or show quantitative consistence with either behavioural or neural data concerning spatial cognition (although qualitative consistence with anatomical and neural findings is pointed out by the authors). Although these models provide explanations, their predictions regarding spatial processing have not been quantitatively evaluated.

Second, in addition to the lack of quantitative validation, their neural implementation is not known, and far from straightforward. For example, implementing the kinds of large matrix inversions and multiplications required by Kalman filters \citep{kalman1960new} is easy on a computer, with centrally coordinated, serial, `fast' computations, but difficult with the kind of distributed, parallel, `slow' (on the level of single neurons, which only spike up to a few dozen times per second) computation performed by the brain. In the domain of world-centered, navigation-scale spatial mechanisms, any suggested neural implementation has to conform with not only the limitations imposed by biological neural networks, but also with the specific connectivity and activity observed in the hippocampal complex, in order to be considered biologically plausible.

In addition to such normative models, a number of mechanistic (implementation-level) models of how uncertainty and inference could be implemented in brains have also been proposed. They can be roughly grouped into three categories - see \citep{pouget2013probabilistic, vilares2011bayesian} for reviews. We briefly summarize these groups below, together with their consistency with what is known about the hippocampus. 
%, which would be suitable for spatial localization or map learning in the face of uncertainty, together with the main reason we chose not to adopt that mechanism in this work. Detailed arguments regarding these reasons can be found in Appendix A.

% free-form approximations do not scale

% no electrophysiological or psychophysical evidence to suggest that the brain can encode multimodal approximations:

\begin{itemize}

\item Probabilistic population codes (PPC) \citep{ma2006bayesian} encode probability distributions in the logarithmic domain by means of a set of coefficients of corresponding exponential basis functions, each coefficient encoded by the activity (spike count) of a neuron. They assume neural variability is independent and Poisson-distributed. However, hippocampal neurons exhibit more variability than a Poisson process \citep{fenton1998place, barbieri2001construction}. Also, if Bayesian inference were implemented in the hippocampus via a PPC, the encoded probability distributions would strongly depend on the firing rate of hippocampal neurons: increased firing rates should mean decreased levels of uncertainty. But empirically, this is not the case - for example, firing rates increase with movement speed \citep{maurer2005self}, which would mean the lowest uncertainties when running fastest (however, faster movements are harder to control and should thus lead to higher uncertainty). 

\item Instead of an encoding in the logarithmic domain, codes in which firing rates are proportional to probabilities have also been proposed, e.g. by \citet{koechlin1999bayesian, barber2003neural}. The problem with their implementation in hippocampal neurons is that the firing rates of these neurons are also influenced by factors unrelated to probability, such as where the animal is headed \citep{ferbinteanu2003prospective} or trial dependent features \citep{allen2012hippocampal}, and can change substantially if either the shape or colour of an environment is altered \citep{leutgeb2005independent}. These influences would strongly interfere with the outcome of the Bayesian inference, if it were implemented in a code that directly utilizes firing rates.

\item Sampling-based codes represent probability distributions with a set of samples drawn from them \citep{fiser2010statistically}. They are asymptotically correct with infinitely many samples, and approximations otherwise. Apart from being able to represent complex, multi-modal distributions, not having to rely on any fixed-form parametrization such as Gaussians, this also allows reducing their accuracy and computational demands by restricting the number of samples used. This property has been used e.g. by \citep{shi2010exemplar} to explain the deviations from the statistical optimum in an exemplar model of a reproduction task. It is difficult to make a general statement as to the implementability of this class of models in the hippocampal complex, as there is a wide variety of suggested concrete neural implementations in non-spatial domains (\citet{sanborn2015types} provides a review), and some applied to navigation space, e.g. \citep{fox2010hippocampus, cheung2012maintaining}. None of them have been quantitatively validated by neural (electrophysiological) measurements, although most of them are supported by behavioural observations. 

%\item \citet{deneve2007optimal} [attractor networks] %  rather than coding for the log probability that a feature is present, neurons code for the log probability that a feature takes on a particular value
%\item \citet{ma2006bayesian} [PPC]
%\item \citet{fiser2010statistically} [sampling]
%\item \citet{friston2011action} [FEP]

\end{itemize}

How the brain might encode and utilize uncertainty is still an open question \citep{pouget2013probabilistic}, but based on the observations regarding the hippocampus outlined above, we argue that a sampling-based code is most suitable in this brain area; in terms of violating as few empirical observations as possible. We will provide electrophysiological evidence of Bayesian inference from single neurons, and propose a possible sampling-based mechanism, in Chapter \ref{cha:bayespc} (and in more detail in Appendix \ref{apx:bayespc}).


%bousquet et al kalman filter

%fox prescott

%penny pioneered ... extend this line of research... difficult to actually implement, as EKF O(n^2)

%Khamassi and Humphries [18] argue that, due to the shared underlying neuroanatomy, spatial navigation strategies that were previously described as being either place-driven or cue- driven are better thought of as being model-based versus model- free. Daw et al. [15] propose that arbitration between model-based and model-free controllers is based on the relative uncertainty of the decisions

% friston action understanding

% make things as simple as possible, but not simpler

% implicitly - RatSLAM

\section{Hypotheses}
\label{sec:intro:hypotheses}

To achieve goals in a spatially extended, realistic environment, at a minimum, an agent (e.g. a biological agent such as an animal, or an artificial agent such as a robot) must be able to 1) move, and keep track of its movements, 2) sense, and interpret its sensations, 3) represent spatial locations in its environment, e.g. of itself and its goal, 4) update these representations when changes occur in the environment, and 5) utilize these representations to achieve its goals (e.g. navigate to its goal location, avoiding dangers). Extensive work on all levels of analysis has been carried out for 1)-3), with the most recent Nobel prize in physiology or medicine awarded on the topic of 3) to John O'Keefe, May-Britt Moser and Edvard I Moser for the discovery of \textit{`cells that constitute a positioning system in the brain'} \citep{burgess2014nobel}. Specifically, it was awarded for the discovery of `place cells' in the hippocampus (which show increased firing in a specific area in the environment, called its `place field'), and of `grid cells' which show a regular, grid-like firing pattern (see Chapter \ref{cha:nnreview} below). 

We have argued above that despite of the variety of existing models regarding 4)+5), new computational models are needed to move towards biological and psychological plausibility as well as real-world capability at the same time (since biological cognition has been shaped by the constraints and challenges of the real world, these should not be neglected in models of cognition). In particular, in accordance with the `Bayesian brain' \citep{knill2004bayesian} and `Bayesian cognition' \citep{chater2010bayesian} paradigms, we have suggested approximate Bayesian inference to be a well-suited mechanism for tackling these challenges. Models on Marr's algorithmic (and implementation) level which utilize such a mechanism require a number of underlying assumptions, some of which can be stated and evaluated as hypotheses. 

We summarize major hypotheses in one place in Table \ref{tbl:hyp} below, and expand on them in the respective results chapters below. The first two concern the representation and manipulation of uncertainty in the hippocampus (required for maintaining approximately accurate location estimates despite noisy sensors and accumulating errors). Hypothesis 3 is needed since unless all remembered landmark locations are corrected

\setlength\tabcolsep{4pt}
%\small
\begin{longtable}{|p{4.5cm}|p{4.5cm}|p{4.6cm}|}
	\hline
	\textbf{Hypothesis} & \textbf{Prediction} & \textbf{Empirical support} \\
	\hline
	
	1 Hippocampal place cells can perform approximate Bayesian inference & {Place field size depends on uncertainty (e.g. proximity of landmarks) in a} & {Place field sizes (recorded from hippocampal neurons of behaving rats) are cor-} \\ \cline{1-1} 
	2 Spatial uncertainty is represented as the size of place cell firing fields & Bayesian fashion & related with uncertainties predicted by a Bayesian model (Chapter \ref{cha:bayespc}) \\ \hline
	3 When revisiting a place, estimates of recently traversed locations and encountered landmarks are updated in an approx. Bayes-optimal fashion & After revisiting parts of an environment, place fields should shift, and recently active place cells should re-activate. Errors should conform to Bayesian predictions & Neural: none in this work, but place fields seem to shift after revisits \citep{mehta2000experience}, and recently active place cells do reactivate (`replay') \citep{carr2011hippocampal}. Behavioural: errors correlate with predictions (Chapter \ref{cha:lida}) \\ \hline 
	4 The structure of spatial representations arises from clustering & Landmarks which are co-represented (belong together) in participants' & Neural: none in this work. Behavioural: the probability of two landmarks being \\ \cline{1-1}
	5 This clustering mechanism operates on features including Euclidean distance, path distance, boundaries, visual and functional similarity & spatial memory should be closer in these features than those not belonging together & co-represented is strongly correlated with distances along these specific features. These distances allow prediction of participant representation structure (Chapter \ref{cha:structure}) \\
	
	\hline
	\captionsetup{width=\textwidth}
	\caption[Hypotheses of the models presented in this work]{\textbf{Hypotheses of the models presented in this work, and empirical support}. Place cell electrophysiological recording data was acquired outside this PhD. All other data has been collected by the author, unless otherwise specified.}
	\label{tbl:hyp}
\end{longtable}


\noindent at every moment (which would likely be intractable), a discrepancy between remembered and actual locations might arise when revisiting a location encountered previously (when traversing a `loop' in the environment). This discrepancy necessitates a backward correction of multiple recent self and landmark locations to maintain consistent representations. The last two are needed to formulate a computational mechanism of spatial representation structure. Structured, hierarchical representations provide clear computational advantages, such as increased speed and efficiency of retrieval search, and economical storage. However, although strong neural \citep{derdikman2010manifold} and behavioural \citep{hirtle1985evidence, mcnamara1989subjective, greenauer2010micro} evidence exists for such structure, underlying computational principles have remained largely unknown.

\section{Outline and Contributions}
\label{sec:intro:outline}

This thesis is presented in the Alternative Format allowed by University of Manchester policy \footnote{http://documents.manchester.ac.uk/DocuInfo.aspx?DocID=7420}, which allows incorporating sections in a format suitable for publication in peer-reviewed journals. We chose the alternative format to more easily accommodate already published work, to reduce risks of self-plagiarism,  and because of the largely self-contained nature of our individual results chapters. Thus, in what follows, the literature review (Chapter \ref{cha:nnreview}) and the three chapters (\ref{cha:bayespc}-\ref{cha:lida}) reporting the results, are copies of papers either accepted by or submitted to peer-reviewed journals. The following list summarizes these papers and the contributions\footnote{In all publications, Madl wrote the draft of the paper, developed the software, designed the experiments, recruited and tested the participants where applicable, and analysed the data. Corrections suggested by Chen, Montaldi, and Franklin were incorporated into the final drafts by Madl after discussions with these co-authors. All publications were supervised by Chen and Montaldi, with Chen mainly commenting on mathematical and computational issues, and Montaldi on psychological and neuroscientific issues.} therein:

\begin{itemize}
	\item Chapter \ref{cha:nnreview}: Madl T., Chen K., Montaldi D. \& Trappl R., 2015. Computational cognitive models of spatial memory in navigation space: A review. \textit{Neural Networks, 65, 18-43.}
	\\ Contributions: 1) a systematic review of representative cognitive models concerned with navigation-scale spatial memory, falling into symbolic, neural network, or cognitive architecture models, including a comparative table of the characteristics of these models.
	\item Chapter \ref{cha:bayespc}: Madl T., Franklin S., Chen K., Montaldi D. \& Trappl R., 2014. Bayesian Integration of Information in Hippocampal Place Cells. \textit{PLoS ONE 9(3), e89762}
	\\ Contributions: 2) first quantitative electrophysiological validation of the representation of spatial uncertainty in the brain, and of Bayesian integration of spatial information in the brain, in three different environments (using data acquired outside this PhD). 3) Formulation and empirical support for an inference mechanism based on coincidence detection (falling into the camp of sampling-based models of neural inference)
	\item Chapter \ref{cha:structure}: Madl T., Franklin S., Chen K., Trappl R. \& Montaldi D., submitted. Exploring the structure of spatial representations. \textit{Cognitive Processing}
	\\ Contributions: 4) behavioural evidence for clustering as the normative principle underlying spatial representation structure, and 5) the first computational model of navigation-scale spatial representation structure on the individual level (able to predict this structure in participants' long-term spatial memory from the geospatial properties of an environment)
	\item Chapter \ref{cha:lida}: Madl T, Franklin S, Chen K, Montaldi D \& Trappl R, submitted. Towards real-world capable spatial memory in the LIDA\footnote{LIDA stands for Learning Intelligent Distribution Agent, and is reviewed in a paper co-authored during this PhD but not included in this thesis: \citep{franklin2013lida}} cognitive architecture. \textit{Biologically Inspired Cognitive Architectures}
	\\ Contributions: 6) integration of three spatial mechanisms capable of dealing with uncertainty and noise into a comprehensive cognitive architecture (localization, map structuring, map correction), and 7) embodying this architecture on a robot, allowing demonstration of the model functionality in a realistic robotic simulator. 8) Proposal of a biologically plausible mechanism for correcting errors in learned maps when revisiting an already known place (the `loop closure' problem, well known in robotics, but neglected in cognitive science), and evaluation against behaviour data regarding cognitive map accuracy in human subjects.
\end{itemize}

The model best accounting for spatial memory structure presented in Chapter \ref{cha:structure} also constitutes a novel kind of metric learning in machine learning, based on the idea of learning a similarity function in the space of absolute pairwise differences (as opposed to e.g. a Mahalanobis distance function). Although proposed before in a similar form for person re-identification in the computer vision community \citep{zheng2011person}, the insight that this space contains neglected information which can be utilized to improve performance in general (not just on image data), and the general formulation allowing arbitrary constituent models for learning a metric in this space, are a novel contribution (9). Since it is too far from the topic of this thesis, metric learning in absolute pairwise difference space is only described briefly (to the extent required to model cognitive map structure) in Chapter \ref{cha:structure}. Applications and results on other kinds of data are briefly summarized in Appendix \ref{apx:adsmetric}.

Before presenting the papers constituting the literature review and results chapters, we summarize the major computational methods employed in this PhD in Chapter \ref{cha:methods} (they are also described in the respective results chapters). After the computational methods, we present the literature review (Chapter \ref{cha:nnreview}) and results (Chapter \ref{cha:bayespc}-\ref{cha:lida}) in the form of published or submitted papers. Subsequently, we continue to discuss the implications of our results, the neural implementability of these mechanisms, and the shortcomings and limitations of our models in Chapter \ref{cha:discussion}. We conclude in Chapter \ref{cha:conclusion} with a conclusion and an outline of potential future work opened up by this research. 

We note that the line of criticism mentioned regarding the neural implementability of the high-level probabilistic models of localization in the previous section also apply to our proposed mechanism of cognitive map structuring (Chapter \ref{cha:structure}). Although it is intended to be a cognitive and not a neural model, we have argued that consistency with the underlying neuroscience can and should play a role in constraining the space of possible models, and evaluating models, even on the algorithmic level. But the map structuring mechanism in Chapter \ref{cha:structure} is, to our knowledge, the first formal model of the observed structure in cognitive maps, both on Marr's computational and algorithmic levels. We did not have the time and resources to extend it down to include a plausible neural implementation within this PhD.

Finally, work done during this PhD has resulted in publicly available software for in-browser 3D spatial memory experiments\footnote{https://github.com/tmadl/Cognitive-Map-Structure-Experiment}, and has contributed to two more publications which are not included in this thesis (the former because it is a conference paper, whereas University policy requires alternative format theses to contain journal papers instead; and the latter because it does not fit in well with the main topic):

\begin{itemize}
	\item Madl T., Franklin S., Chen K. \& Trappl R., 2013. Spatial working memory in the LIDA cognitive architecture. \textit{Proceedings of the International Conference on Cognitive Modelling (2013), pp. 384-389}
	\item Franklin S., Madl T., D'Mello, S., Snaider, J., 2014. LIDA: A Systems-level Architecture for Cognition, Emotion, and Learning. \textit{IEEE Transactions on Autonomous Mental Development 6(1), pp. 19-41}
\end{itemize}

\nocite{madl2015computational, madl2014bayesian, madlexploring, madltowards, madl2013spatial, franklin2013lida}

\chapter{Computational Methods}
\label{cha:methods}

This section briefly reviews the computational methods employed in this thesis. Figure \ref{fig:methods} shows an overview over all employed methods, and the way they are utilized to support the mechanisms, algorithms, and cognitive models presented below. More specific and detailed descriptions of each method, including the empirical methods used to gather data for testing hypotheses for which no existing datasets could be found, are described in the respective result chapters. Chapter \ref{cha:bayespc} contains descriptions of the place cell firing data and its computational modelling, and Chapter \ref{cha:structure} the collection of data regarding spatial memory structure and accuracy data and its modelling. Chapter \ref{cha:lida} integrates the models in the same architecture, and interfaces them to a robotic simulator. 
Describing the details of the LIDA (Learning Intelligent Distribution Agent) cognitive architecture \citep{franklin2013lida}, within which these cognitively plausible Bayesian mechanisms have been computationally realized, is outside the scope of this thesis.

\begin{figure}[h]
	\centering
	\includegraphics[width=\textwidth]{img/methodsfigure2}
	\caption[Overview of how the methods in this thesis help support real-world capable models of cognition]{\textbf{Overview of how the methods in this thesis help support real-world capable models of cognition,} roughly divided into empirical methods (bottom half) and computational methods (top half). Gray boxes contain data/code used to substantiate or implement some models, but not gathered/implemented by us.} 
	\label{fig:methods}
\end{figure}

\section{Probabilistic modelling}

Probabilistic models use probability distributions to represent quantities and the uncertainties associated with them, utilizing probability theory to manipulate these distributions \citep{ghahramani2015probabilistic}. Two basic rules provide the foundation, and together yield Bayes' theorem, which underlies Bayesian modelling. The \textit{sum rule} takes the form

\begin{equation}
\label{sumrule}
p(Y) = \sum_{X} p(Y, X),
\end{equation}

where $p(X,Y)$ is the joint probability (i.e. the probability of X and Y) and the summation is over all values which $Y$ could possibly take. $p(X)$ is also referred to as the marginal probability, and the summation in Equation \ref{sumrule} is also called marginalization (which is especially useful to make inferences about variables of interest by summing out all other variables). The \textit{product rule} states that

\begin{equation}
\label{productrule}
p(Y,X) = p(Y|X)p(X) = p(X|Y)p(Y),
\end{equation}

where $p(Y|X)$ is the conditional probability (i.e. the probability of Y given X). Combined, they yield \textit{Bayes' theorem}:

\begin{equation}
\label{bayesrule}
p(Y|X) = \frac{p(X|Y)p(Y)}{p(X)} = \frac{p(X|Y)p(Y)}{\sum_{Y} p(X, Y)}.
\end{equation}

In the context of a probabilistic model, defined by a number of parameters encoded in $Y$, and given some observed data encoded in $X$, we can use Equation \ref{bayesrule} to calculate a \textit{posterior} probability distribution of model parameters, combining \textit{prior} knowledge (or assumptions) $p(Y)$ with the \textit{likelihood} $p(X|Y)$.

The sections below summarize computational-level solutions to the problems required for real-world spatial cognition outlined in Chapter \ref{cha:intro} in this probabilistic framework. As mentioned there, the goal of this work is contributing to the understanding of spatial information processing in brains and minds, and not finding particularly good solutions to these problems. Numerous algorithms capable of much more accurate localization and mapping and making less restrictive assumptions have been proposed in probabilistic robotics \citep{thrun2005probabilistic}, more specifically simultaneous localization and mapping (SLAM) - see \citep{thrun2008simultaneous,durrant2006simultaneous,bailey2006simultaneous} for reviews and \citep{tuna2012evaluations} for a more recent evaluation. 

Our particular computational-level solutions for estimating locations are not novel, and utilize stronger simplifications compared to the state of the art in SLAM. We are applying existing computational and mathematical tools to cognitive and neural mechanisms, following a long and successful history of this approach in the field of computational cognitive modelling \citep{sun2008introduction}, which can be seen as a branch of applied computer science. In this field, simplicity and approximations can be assets; since humans are unlikely to use computationally complex, optimal statistical models (see e.g. \citep{van2008tractable,simon1955behavioral}). A simpler, sub-optimal model which nevertheless explains empirical data better, and is more consistent with neural anatomy, is better suited to modelling cognition than an intractable or implausible optimal model. Furthermore, the implementation of these abstract methods in a way consistent with the neuroscience and psychology of spatial memory is novel, as is their integration with a comprehensive cognitive architecture and their substantiation with empirical data (see Section \ref{sec:intro:outline} for the full list of novel contributions). 

\section{Bayesian cue integration}

One concrete application of Equation \ref{bayesrule} is the inference of the most likely current location of an animal, given some observations regarding the distance of a number of landmarks. For simplicity, we assume 1) a uniform prior over these observations, and 2) conditional independence of the observations given the location. The posterior probability of the current location $p(\bm x | \bm O)$, given a location prior $p(\bm x)$ and some observations $o_1, ..., o_N \in \bm O$, is 

\begin{equation}\label{bayes1}
p( \bm x | \bm O ) = \frac{p( \bm x ) p( \bm O | \bm x )}{p(\bm O)} = \gamma p( \bm x ) p( \bm O | \bm x )
\end{equation}

The prior can be obtained by adding up self-motion signals (a process called `path integration' or dead reckoning - see Chapter \ref{cha:nnreview}). Individual observation distributions can express distance measurements to landmarks, and can be multiplied due to their conditional independence:

\begin{equation}\label{bayes2}
p( \bm x | \bm O ) = \gamma p( \bm x ) \prod_{i=1}^{N} p( \bm o_i | \bm x ).
\end{equation}

For now, we further assume that each of these variables is normally distributed. This makes it straightforward to derive the variance $C_L$ of the normal/Gaussian posterior location distribution $p( \bm x | \bm O ) = \mathcal{N}(\bm x ; \mu_L, C_L)$ from the variances of the prior and of the likelihood distributions $C_x$ and $C_{o,i}$ (see e.g. \cite{wu2004properties} for the derivation of the parameters of products of Gaussian distributions):

\begin{equation}\label{bayes3}
C_{P}=(C_x^{-1}+\sum_{i=1}^{N} C_{o,i}^{-1})^{-1}
\end{equation}

In the one-dimensional case, the variance is the square of the standard deviation $\sigma$. We can say that the standard deviation of a Gaussian distribution is a measure of the `uncertainty' associated with it (as it measures the spread among possible values - the more certainly a value is known, the lower the associated $\sigma$ of the distribution describing it). Assuming that the observation uncertainties $\sigma_{o,i}$ depend linearly on the respective distances $d_{i}$, such that $\sigma_{o,i}=s \cdot d_{i}$ (Chapter \ref{cha:bayespc} provides justifications and evidence for this linear relationship), we obtain the standard deviation of the location posterior for a given set of measurement distances:

\begin{equation}\label{bayes4}
\sigma_{P}(d_1, ..., d_N)=\sqrt{(\sigma_x^{-2}+s \sum_{i=1}^{N} d_{i}^{-2})^{-1}}.
\end{equation}

Chapter \ref{cha:bayespc} uses Equation \ref{bayes4} to test the hypotheses that place cells may represent uncertainty and perform Bayesian cue integration. Although place cells constitute a two-dimensional representation, this one-dimensional treatment of observation likelihoods is an acceptable approximation in the kinds of environments from which the data was collected (rectangular boxes without landmarks, where the axes can be assumed to be independent as they are orthogonal, and a very narrow, circular track with landmarks, where the width can be neglected as it is less than $3\%$ of the length). 



%Figure \ref{fig:bayescue}

%\begin{table*}[h]
%	\centering
%	{\renewcommand{\arraystretch}{1.2}
%		\begin{tabu}{c|c}
%			$\downarrow$ {Dimensions} & {Variance of the location posterior}\\ \tabucline[3pt]{-}
%			1D & $\sigma^2_{L1D}=(\frac{1}{\sigma_x^2}+\sum_{i=1}^{N} \frac{1}{\sigma_{o,i}^2})^{-1}$ \\
%			2D & $C_{L2D}=(C_x^{-1}+\sum_{i=1}^{N} C_{o,i}^{-1})^{-1}$ \\
%		\end{tabu}
%	}
%	\caption[Variance of the posterior location estimate under Gaussian assumptions]{\textbf{Variance of the posterior location estimate under Gaussian assumptions}. $\sigma$ stands for scalar standard deviations, and $C \in \mathbb{R}^{2x2}$ for covariance matrices.}
%	\label{tbl:bayescue}
%\end{table*}

\begin{figure}[h]
	\centering
	\includegraphics[width=\textwidth]{img/bayesian_localization3}
	\caption[Bayesian cue integration for localization]{\textbf{Bayesian cue integration for localization.} Illustration of how an animal might use its prior location belief (blue) estimated from its movement, and distance distributions e.g. to a boundary (green) to obtain a corrected location estimate (red) using Bayesian inference.} 
	\label{fig:bayescue} 
\end{figure}

%
\section{Bayesian localization}

To maintain a location estimate through time, the kind of cue integration described above has to be performed regularly (after every time step). One source of location information is adding up each movement vector, a process called odometry in robotics and `path integration' in cognitive science and biology. However, movements are not accurate and noise free in real-world environments - each movement vector contains a slight error, and these errors add up over time. Eventually, these accumulating errors render the location estimate useless, if sensory information is not used to correct it. 

Bayesian localization is concerned with correcting the location estimate in time using noisy observations \citep{thrun2005probabilistic}. Conceptually, it entails performing the Bayesian cue integration to correct location estimates \textit{recursively}, after every movement / time step. Its operation can be summarized in three stages, which are performed iteratively at every time step: 1) movement (adding the current movement), 2) correction of the location estimate via Bayesian cue integration, 3) updating of the path integration estimate for use in the next iteration.

Unlike the simplified treatment above, which has considered only one snapshot in time, Bayesian localization considers the posterior at any time step $t$. This posterior distribution has to depend on all movements until now: $\bm m_{1:t}$, on all observations until now: $ O_{1:t}$, as well as the locations of known landmarks $\bm l_{1:N}$. Extended by these dependencies, the posterior location distribution from Equation \ref{bayes1} becomes

%specifying the probability distribution of the current position $ x_{t} $ given motor commands $u$, measurements $z$, and landmarks $l$. This expression can be expanded using Bayes' rule,

\begin{equation}
\label{bayesloc1}
p(\bm x_{t} | \bm m_{1:t}, \bm O_{1:t}, \bm l_{1:N}) = \gamma p(\bm O_{t} | \bm x_{t}, \bm l_{1:N}) p(\bm x_{t} | \bm m_{1:t}),
\end{equation}

through simple application of Bayes' theorem. We can use the sum rule (with the sum replaced by an integral for dealing with continuous distributions) to model the `path integration' (odometry) mechanism which provides the prior in Equation \ref{bayesloc1}:

\begin{equation}
\label{prior}
p(\bm x_{t} | \bm m_{1:t}) = \int p(\bm x_{t} | \bm x_{t-1}, \bm m_{t-1}) p(\bm x_{t-1} | \bm m_{1:t-1}) \mathrm{d} \bm x_{t-1}.
\end{equation}

This equation allows inferring the current location prior based on the most recent movement $m_{t-1}$ and on the previous location estimate $\bm x_{t-1}$ by marginalizing (integrating out) the previous location. This is a recursive formulation which yields a path integration estimate based on a starting location and a number of movements. This estimate is subject to accumulating errors. However, crucially, the corrected previous location estimate (previous posterior) can be used instead of the uncorrected previous path integration estimate. Using this insight, replacing $p(\bm x_{t-1} | \bm m_{1:t-1})$ in Equation \ref{prior} by the previous location posterior $p(\bm x_{t-1} | \bm m_{1:t-1}, \bm O_{1:t-1}, \bm l_{1:N})$ and plugging the resulting prior into Equation \ref{bayesloc1} yields

\begin{equation}
\label{bayeslocsolution}
\begin{split}
p(\bm x_{t} | \bm m_{1:t}, \bm O_{1:t}, \bm l_{1:N}) = \gamma p(\bm O_{t} | \bm x_{t}, \bm l_{1:N}) \int p(\bm x_{t} | \bm x_{t-1}, \bm m_{t-1}) \cdot \\
 p(\bm x_{t-1} | \bm m_{1:t-1}, \bm O_{1:t-1}, \bm l_{1:N}) \mathrm{d} \bm x_{t-1}
\end{split}
\end{equation}

This recursive equation for updating location estimates is a Bayes-optimal solution to the localization problem and allows inferring the current location based on two conditional densities: a model specifying the effect of movements on the location (a `motion model'):
\begin{equation}\label{motionmodelq}
p(\bm x_{t} | \bm x_{t-1}, \bm m_{t-1})
\end{equation}
and a model specifying the probability distribution of the current measurements $ \bm O_{t} $ at a position $ \bm x_{t} $ given the landmarks $ \bm l_{1:N} $ (a `sensor model'):
\begin{equation}\label{sensormodelq}
p(\bm O_{t} | \bm x_{t}, \bm l_{1:N}).
\end{equation}

Equation \ref{bayeslocsolution} is the mathematical formulation of Bayesian localization, which, conceptually, iterates over the three stages mentioned above: movement (application of the motion model), correction (via Bayes' theorem), and update.

As argued in Chapter \ref{cha:bayespc} and Appendix TODO, the activity of hippocampal place cells can be viewed as samples from probability distributions, and the size of their firing fields can be partially predicted by a Bayesian model. We will also argue based on existing evidence that the `motion model' is implemented by a neural path integrator in the entorhinal cortex, and that neurons with boundary-related firing might implement the `sensor model'.

Such a sampling-based representation of uncertainty in these spatially relevant brain areas naturally suggests employing a sequential Monte Carlo method \citep{doucet2000sequential} to computationally evaluate the integral in equation \ref{bayeslocsolution}. Although the usual method of choice in robotics is importance sampling \citep{montemerlo2007fastslam, thrun2005probabilistic}, we approximate the integral using rejection sampling \citep{doucet2000sequential}, and will argue in Chapter \ref{cha:bayespc} and Appendix TODO that coincidence detection in hippocampal place cells can implement this mechanism.

From a computational point of view, instead of inferring the parameters of the location posterior distribution (e.g. the mean and variance in case of a Gaussian), we represent it by sampling multiple location hypotheses. The mean of these hypotheses corresponds to the 'best guess' estimate, and their standard deviation to the associated uncertainty. Apart from the empirical evidence for a sampling based mechanism (see Chapter \ref{cha:bayespc}, as well as \citep{fiser2010statistically} for a more general review), the main advantage of this approach is the ability to represent free-form distributions (irregular, non-Gaussian, multimodal distributions etc.).

Particles (samples, hypotheses) $\bm x^i$ are generated regularly based on self-motion information (linear and angular movement speed $v$) according to the motion model (Equation \ref{motionmodelq}), performing path integration - in the simplest case: $ \bm x_{t}^i = \overline{\bm x}_{t-1} + \bm v'\Delta t $ - at simulated timesteps $ \Delta t $. Gaussian noise is multiplied to the estimated speed to obtain a distribution of hypotheses reflecting the path integration / odometry uncertainty (neither animals nor robots can estimate their movement speed with perfect accuracy):

$\bm v' = \bm v_{true} \cdot \mathcal{N}(1, \begin{bmatrix}\sigma_v^2 & 0\\ 0 & \sigma_\omega^2\end{bmatrix}) $,

where $ \sigma_v^2 $ and $ \sigma_\omega^2 $ are model parameters representing the variance in the linear and angular speeds, respectively. Since the estimate of $\bm v$ is noisy, accumulating errors would lead to an increase of uncertainty and the corruption of the distribution represented by the set of particles, which is why correction with the sensor model is required. 

Under Gaussian assumptions, this correction can be implemented simply by multiplying a path integration prior and a number of sensory likelihoods and solving for the means and variances (Equation \ref{bayes2}). The ensuing algorithm for Bayesian localization is trivial. When using samples instead of a Gaussian to represent the posterior, the correction can be implemented by rejection sampling \citep{doucet2000sequential}, i.e. by deleting hypotheses inconsistent with sensory measurements (see Figure \ref{fig:bayesloc}). The derivation of why this rejection sampling scheme approximates the true Bayesian posterior can be found in Appendix TODO. Details regarding how brains could implement this algorithm are discussed in Chapter \ref{cha:bayespc}.

\begin{figure}[h]
	\begin{pseudocode}{movement}{samples, \textbf{v}, N}
		1: prevmean \GETS mean(samples) \\
		2: newsamples \GETS \{ \} \\
		3: \FOREACH particle \in samples \\
		4: \quad newsamples \GETS newsamples \cup {motionModel(particle, \textbf{v})} \\
		5: \WHILE count(newsamples) < N \\
		6: \quad newsamples \GETS newsamples \cup {motionModel(prevmean, \textbf{v})} \\
		7: return(newsamples)
	\end{pseudocode}
	\begin{pseudocode}{correction}{samples, \textbf{O}, \textbf{L}}
		1: newsamples \GETS \{ \} \\
		2: \FOREACH particle \in samples \\
		3: \quad likelihood \GETS sensorModel(particle, \textbf{O}, \textbf{L}) \\
		4: \quad \IF random() < likelihood \\
		5: \quad \quad newsamples \GETS newsamples \cup {particle} \\
		6: return(newsamples)
	\end{pseudocode}
	\begin{pseudocode}{localizationStep}{posteriorsamples, \textbf{v}, \textbf{O}, \textbf{L}, N}
		1: timestep++ \\
		2: movedsamples \GETS movement(posteriorsamples, \textbf{v}, N) \\
		3: correctedsamples \GETS  correction(movedsamples, \textbf{O}, \textbf{L}) \\
		4: return(correctedsamples)
	\end{pseudocode}
	\centering
	\includegraphics[width=0.75\textwidth]{img/rejectionsampling}
	\caption[Bayesian localization algorithm with rejection sampling]{\textbf{Bayesian localization algorithm with rejection sampling,} producing updated posterior samples given the samples from the previous posterior, the speed vector $\bm v$ and observations $\bm O$ at the current time step, landmarks $L$, and a particle budget $N$}
	\label{fig:bayesloc}
\end{figure}

\clearpage

%To correct this distribution, and to estimate the target posterior distribution \eqref{bayeslocsolution}, each particle is weighted according to the likelihood of the perceived sensory measurements from the location hypothesis represented by the particle, i.e. according to the sensor model: 
%$\bm w_t^i=p(\bm O_t|\bm x_t^i, \bm O_{t-1}, \bm m_t)$. This approach of estimating a target distribution is called importance sampling (see e.g. \cite{montemerlo2007fastslam, thrun2005probabilistic} for the derivation).

\section{Maximum likelihood map error correction}

Landmark location estimates can be updated in the same way as the agents' location estimates $\bm x$, by integrating new observations into the posterior distribution representing these locations (either in the form of Gaussians or of samples from this distribution). With infinitely many particles, the algorithm presented in Figure \ref{fig:bayesloc} would suffice to maintain correct location estimates. 

However, there are practical limits on the particle budget (due to limited computational resources in computers, and due to limited firing rates in neurons). This necessarily leads to errors whenever there is no particle at the unobservable true location. Unfortunately, these errors add up as well. They become most pronounced when revisiting an already known part of the environment, i.e. when traversing a loop - although the agent has returned to its starting location, it will think that it is at a new location, and form new representations of the same place. Multiple such loops can lead to multiple redundant, erroneous representations. 

The problem of how to correct spatial representations when revisiting a known place (not only the location estimate but also the estimated recent path and landmark locations) is the `loop closing' problem in robotics (see e.g. \citep{williams2009comparison, thrun2008simultaneous}). Brains need to solve this problem as well - although human spatial representations are not perfectly accurate, humans are able to correct mistaken estimates when they recognize a revisited place. Interestingly, despite the abundant robotics literature on the topic of closing loops, this problem has been largely neglected in cognitive science literature. 

Our cognitive model of loop closing is described in more detail Chapter \ref{cha:lida}. Here, we will briefly summarize its purely computational and mathematical aspects. We will assume that it is sufficient to correct the route taken during the loop. When performing large-scale loop closing, the model in Chapter \ref{cha:lida} applies the same correction to a position and the local landmarks around it (a simplification justified based on neuroscientific evidence in that chapter). We also make the assumption that correction only concerns position representations and not angular representations, once again based on neural evidence. Hippocampal `reverse replay' \citep{carr2011hippocampal} (the re-activation of recently active place cells) is a plausible mechanism for correcting the recent route when revisiting a location, as argued in Chapter \ref{cha:lida}, but such a mechanism has not been found for neurons with direction-specific firing.

When revisiting a known place, the recently traversed path has to be corrected using the discrepancy between the previously and recently estimated location of the revisited place. Naturally, the recent estimate has to be reset to be equivalent to the previous one, but it is not obvious how to correct the other recently visited locations $\bm x_0, ..., \bm x_m \in X$ along the recent path $X$. Let $\bm c_1, ... \bm c_m \in C$ denote a set of vectors, each specifying the difference between two locations $\bm c=\bm x_a-\bm x_b$, and each associated with a measurement uncertainty $S_c$ in the form of the covariance matrix of a normal distribution. For locations traversed in sequence, $\bm c$ and $S_c$ is given by the motion model (by path integration). For revisited locations, $\bm c$ is zero. 

According to Bayes' theorem, and assuming that constraints are independent given the location, the recent path depends on the product of the constraint distributions; and the best path estimate is the one that maximizes:

\begin{equation}
\label{loopprob}
P(X|C) \propto \prod_{i=1}^{m} P(\bm c_i|X) 
\end{equation}

Each $P(\bm c_i|X)$ expresses the likelihood that this constraint is satisfied by the path $X$, as a Gaussian distribution: $P(\bm c_i|X)\propto\mathcal{N}(\bm x_a-\bm x_b;\bm c_i, S_i)$ (where $\bm x_a$ and $\bm x_b$ are the location estimates which should have the distance $\bm c_i$ according to this constraints). We are interested in the maximum of Equation \ref{loopprob}, which is equivalent to the minimum of its negative logarithm. If we define $\bm d_i=\bm x_a - \bm x_b - \bm c_i$ as the discrepancy between the constraint and the locations it concerns within the path, then this solution is given by:

\begin{equation}
\label{mleq}
X_{ml} = \argmax_X P(X|C) = \argmin_X -log P(X|C) = \argmin_X \sum_{i=1}^{m} ||\bm d_i||_{S_i^{-1}}.
\end{equation}

Equation \ref{mleq} mathematically describes the maximum likelihood error correction problem for loop closing. It tries to minimize the discrepancies between the constraints and the estimated locations, taking into account the constraint uncertainties $S_i$ by utilizing the Mahalanobis distance\footnote{The Mahalanobis distance is defined as $||\bm x_1-\bm x_2||_C = \sqrt{(\bm x_1-\bm x_2)^TC(\bm x_1-\bm x_2)}$} to measure the discrepancy.

There are several ways to solve Equation \ref{mleq}. We chose sequential gradient descent because it can be implemented in biological neurons \citep{bengio2015towards}. \citep{olson2006fast} derive this solution, giving the following gradient with respect to constraint $i$, depending on a learning rate $\alpha$, a full Jacobian $J$ of the constraints with respect to the path, and the Jacobian $J_i$ of constraint $i$:

\begin{equation}
\label{gradient}
\Delta X \approx \alpha (JS^{-1}J)^{-1}J_i^TS_i^{-1} \bm d_i.
\end{equation}

Because of the incremental structure of the Jacobian, it is possible to simplify this expression (see Chapter \ref{cha:lida}). Making use of this structure, and defining a loop precision parameter $A_i=S_i/S_P$ specifying the ratio of the uncertainties of loop closure constraints (added when revisiting a place) and path integration constraints, the gradient for each individual location within the loop becomes:

\begin{equation}
\label{correction}
\Delta \bm x_j \approx \alpha d_i \frac{\sum_{k=a+1}^{j} S_i^{-1}}{\sum_{k=a+1}^{min(j,b)} S_P^{-1}} = \alpha A_i \bm d_i p_j,
\end{equation}

where $p_j=(min(j,b_i)-a_i-1)/(b_i-a_i-1)$ denotes how far $x_j$ lies along the loop, with $0 \leq p_j \leq 1$. Unlike usual gradient descent procedures, in this particular case we know that $\Delta \bm x \leq \bm d_i $ must hold, and can prevent the algorithm from overshooting, accelerating its convergence. Figure \ref{fig:sgdslam} contains the algorithm using this gradient to correct location estimates when revisiting a place.

\begin{figure}[h]
	\begin{pseudocode}{correctPath}{X, loopConstraints, \alpha, A, N}
		1: \WHILE i < N \AND \NOT converged \\
		2: \quad i++ \\
		3: \quad \FOREACH a,b \in loopConstraints \\
		4: \quad \quad discrepancy \GETS X_a - X_b \\
		5: \quad \quad \FOREACH j \in (a,b] \\
		6: \quad \quad \quad p \GETS (min(j,b)-a-1)/(b-a-1) \\
		7: \quad \quad \quad \beta \GETS min(\alpha A \cdot discrepancy, discrepancy) \\
		8:\quad\quad\quad X_j \GETS X_j + \beta p \\
		9: return(X)
	\end{pseudocode}
	\caption[Algorithm for correcting location estimates when revisiting places]{\textbf{Algorithm for correcting location estimates when revisiting places (`loop closing'),} producing a corrected path given the estimates of locations $X$ along that path (from Bayesian localization), a list of loop constraints indicating the same (revisited) places (from landmark recognition or place recognition), a learning rate $\alpha$, a loop precision parameter $A$ and an iteration budget $N$}
	\label{fig:sgdslam}
\end{figure}

\section{Bayesian nonparametrics for map structuring}

It has been suggested that map-like spatial representations are structured hierarchically \citep{hirtle1985evidence,mcnamara1989subjective,greenauer2010micro}, but no formal model has been put forth for a process that might account for this structure. We hypothesize in Chapter \ref{cha:structure} that this process might be clustering. Computationally, we chose a Dirichlet Process Gaussian Mixture Model (DP-GMM) to account for the behaviour data we collected (see Chapter \ref{cha:structure}), for two reasons. First, DP-GMMs (unlike most clustering algorithms) are able to infer the number of clusters, not just cluster memberships; and are infinitely extensible \citep{rasmussen1999infinite}. Second, Bayesian nonparametric models with Dirichlet priors have a successful history in psychological modelling, e.g. of category learning and causal learning \citep{tenenbaum2011grow}, transfer learning \citep{canini2010modeling}, and human semi-supervised learning \citep{gibson2013human}.

By `map structure', here and in Chapter \ref{cha:structure}, we mean sub-map memberships. There is evidence that human spatial maps are hierarchical \citep{hirtle1985evidence,mcnamara1989subjective,greenauer2010micro}, just as geographical maps are - e.g. there is a map of the country and a map of the cities therein; and any given building may be represented not only on the country map but also on one of the city maps. Similarly, any object (e.g. building) memorized by a participant belongs to her map-like spatial representation (`cognitive map'), as well as to one of its sub-maps. We only consider a two-level hierarchy (map and sub-maps); thus, sub-map memberships fully describe our modelled map structure.

A number of features can influence spatial representation structure, including spatial distance and visual and functional similarity of landmarks. The importance of these features varies across participants, and these subject-specific importances have to be accounted for before the clustering process. We chose to implement a new metric learning method to do so (see below). Our model of spatial representation structure consists of these two components: a subject-specific metric, expressing the `similarity function' between two buildings, and the DP-GMM model for clustering buildings under this metric.

As noted in the Introduction, unlike the rest of our work, we have not shown what the neural implementation of such a structuring process might look like. Some prior work exists showing the possibility of inference in hierarchical Bayesian models such as the DP-GMM, e.g. \citep{shi2009neural} - see \citep{sanborn2015types} for a review. We have substantiated the psychological plausibility of this model by showing that it can explain and predict human behavior data (Chapter \ref{cha:structure}), and leave the investigation of the biological plausibility of this specific mechanism for future work.

\subsection{Dirichlet Process Gaussian Mixture Models for clustering}

We will only describe the DP-GMM model very briefly, since it is a well-established model and since we did not implement it ourselves in this work (we used the $bnpy$ Python library instead). See e.g. \citep{rasmussen1999infinite} for its introduction, or \citep{gershman2012tutorial} for a tutorial. The DP-GMM model can be defined as follows:

\begin{equation}
\label{eq:dpgmm}
\begin{array}{rcl}
\phi_k   \sim Beta(1, \alpha_1) \\
\mu_k    \sim Normal(0,  \mathbf{I}) \\
\Sigma_k \sim Wishart(D, \mathbf{I}) \\
\pi_{k}  \sim SBP(\phi) \\
x_t \sim Normal(\mu_{z_i},  \Sigma_{z,i}^{-1}),
\end{array}
\end{equation}

where SBP stands for the stick-breaking process for generating mixture weights: $\pi_k=v_k \prod_{j=1}^{k-1} (1-v_j)$. Data can be generated from this model by first choosing a cluster with probabilities specified by mixture weights: $z \sim Cat(\pi)$, and then drawing an observation from the parameters of that cluster $x \sim Normal(\mu_z, \Sigma_z)$.

Given the data, the parameters of this model can be inferred using either a Monte Carlo chain sampling method \citep{neal2000markov} or variational inference \citep{blei2006variational}. We did not implement an inference algorithm in this work; instead, we have used the $bnpy$ Python library for this purpose. See \citep{hughes2013memoized} for implementation details.

\subsection{Metric learning in absolute pairwise difference space}

In order to learn a suitable metric for our data, we had to develop a novel metric learning method, since the assumptions made by existing methods do not hold in our case. Neither the linear separability assumption (made by linear metric learning), nor the prerequisite of roughly isotropic variances along the features (made by RBF-based methods \citep{ong2005learning}) is the case for all subjects in our dataset. Furthermore, a GDA-based metric can naturally incorporate the hypothesis that same sub-map building pair differences should be located close to the origin, and should be separable from different-map building pairs (these two distributions of pair differences can be naturally modelled using Gaussian distributions) - see Chapter \ref{cha:structure}. 

Our proposed method can be seen as a novel approach to perform non-linear metric learning using weak supervision in the form of pairwise constraints, in order to improve clustering performance, as pioneered by \cite{xing2002distance}. The problem to be solved can be defined as follows. Let $\mathcal{X}=(\boldsymbol x_i, ..., \boldsymbol x_n)$ be the feature vector representation of $n$ objects which are to be clustered, where $x_i \in \mathbb{R}^D$ are vectors with $D$ dimensions. Let the set of $m$ given labelled pairwise co-representation constraints be denoted by $\mathcal{C}$, where $ \lvert \mathcal{C} \lvert = m $, and $c_{i,j} \in \mathcal{C}$ is

\begin{equation}
c_{i,j}=
\begin{cases}
1, & \text{if $i$ and $j$ belong to the same sub-map (co-represented)} \\
0, & \text{if $i$ and $j$ belong to different sub-maps (not co-represented)}
\end{cases}
\end{equation}

Our ultimate goal is to group the $n$ objects into $K$ clusters (`sub-maps'), such that objects of the same cluster are more similar to each other than to those of different clusters; taking into account the provided pairwise constraints to learn a good similarity metric for the given data.

Conventional approaches leveraging non-linear metric learning for this problem try to find a kernel $\Phi$ such that the clustering resulting from using the distance metric defined by that kernel, $d^2(x_1, x_2)=(\Phi(x_1)-\Phi(x_2))^T(\Phi(x_1)-\Phi(x_2))$, does not violate the provided constraints (ensures co-represented pairs are closer than other pairs, if possible), and often employ RBF kernels for this purpose, e.g. \citep{baghshah2010kernel, chitta2011approximate}. 

In contrast, the proposed framework aims to learn the distribution of co-representation probabilities (whether or not two object should be linked) from the provided set of constraints, and constructs a pseudo-metric based on a generative model of co-representation probabilities. Crucially, this probabilistic model is defined on the vector space of absolute pairwise differences (APD), which allows learning the importance of each feature (a challenge for RBF kernels for data with non-isotropic variance). Learning in APD space has been proposed before by \cite{zheng2011person} (specifically for person re-identification in computer vision), but not as a general metric learning method. The metric based on this generative model is a pseudo-metric, because it does not satisfy the conditions of subadditivity, $d(x, z) \leq d(x, y)+d(y,z)$ and the identity of discernibles, $d(x, y) = 0 \enskip \text{if and only if} \enskip x = y$.

Let $[\Delta \boldsymbol{x_{i,j}}]_+ = \big( \lvert \boldsymbol{x_{i,k}} - \boldsymbol{x_{j,k}} \lvert \big)_{k=1}^m $ be the representation of each pair of objects $(i,j)$ in APD vector space. The co-representation probability distribution, i.e. the posterior probability of any pair of objects belonging to the same cluster, given a pair of objects and some model parameters $\boldsymbol{\theta}$ is then 

\begin{equation}
\label{eq:linkprob}
p(c=1|\Delta \boldsymbol{x}, \boldsymbol{\theta}) \propto p(\Delta \boldsymbol{x} | c=1, \boldsymbol{\theta}) p(c=1|\boldsymbol{\theta})
\end{equation}

The likelihood $ p(c=1|\Delta \boldsymbol{x}, \boldsymbol{\theta}) $, the model parameters $ \boldsymbol{\theta} $ (as well as the prior) can be estimated from $\mathcal{X}$ and $\mathcal{C}$, even in closed form, using Gaussian Discriminant Analysis (GDA). This yields a suitable non-linear pseudo-metric based on this probability distribution - see Equation \ref{eq:metric} -, such that objects likely to belong to the same cluster will be close, and those likely to belong to different clusters will be far apart; with these distances directly depending on co-representation probabilities. 

\begin{equation}
\label{eq:metric}
d_m(\boldsymbol x_1, \boldsymbol x_2; \boldsymbol{\theta}) = 1 - p(c=1|\Delta \boldsymbol x, \boldsymbol{\theta}) = p(c=0|\Delta \boldsymbol x, \boldsymbol{\theta})
\end{equation}

A metric is well-suited for clustering if within-cluster instances are closer than across-cluster instances according to it; i.e. if for any $\Delta \boldsymbol x_m \in \mathcal{M}, \Delta \boldsymbol x_c \in \mathcal{C} $ it holds that $ d_m(\boldsymbol x_{m,1}, \textbf x_{m,2}; \boldsymbol{\theta}) < d_m(\boldsymbol x_{c,1}, \textbf x_{c,2}; \boldsymbol{\theta}) $. It follows from Equation \ref{eq:metric} that this is the case if the generative model learns to optimally separate the absolute differences of within-cluster instance pairs from across-cluster pairs.

In the generative \textbf{GDA} model \citep{bensmail1996regularized}, the likelihoods of a pair of instances either being co-represented (i.e. belonging to the same sub-map), or not being co-represented (i.e. belonging to different sub-maps) are each modelled using a multivariate Gaussian: 

\begin{equation}
\label{eq:mvnormal}
p( \Delta \textbf x | c=i; \mu_i, \Sigma_i) = (2\pi)^{-\frac{D}{2}}|\boldsymbol\Sigma_i|^{-\frac{1}{2}}\, e^{ -\frac{1}{2}(\Delta \mathbf{x}-\boldsymbol\mu_i)^\intercal\boldsymbol\Sigma_i^{-1}(\Delta \mathbf{x}-\boldsymbol\mu_i) },
\end{equation}

where $i \in \{0,1\}$. $(\mu_1, \Sigma_1)$ are the means and covariances of the APD distances of co-represented pairs, and $(\mu_0, \Sigma_0)$ those of not co-represented pairs. These parameters can be easily estimated from the two given sets of co-represented and not co-represented object pairs, respectively, by calculating their means and covariances.

From Equation \ref{eq:mvnormal} and Bayes' theorem, we obtain the generative probability required for the metric in \ref{eq:metric}, which then becomes:

\begin{equation}
\label{eq:adsgda}
d_m(\boldsymbol x_1, \boldsymbol x_2; \boldsymbol{\theta}) = 1 - \frac{p( \Delta \textbf x | c=1; \mu_1, \Sigma_1)}{\sum_{i \in \{0,1\}} {p( \Delta \textbf x | c=i; \mu_i, \Sigma_i)}}
\end{equation}

Thus, the trained GDA-model can be used to calculate distances (Equation \ref{eq:adsgda}) between all pairs of objects in any testing data set. The data is projected under this metric using distance-preserving embedding. We have used multi-dimensional scaling (MDS) for this purpose \citep{borg2005modern}. The result of this projection is a data set embedded such that Euclidean pairwise distances therein reflect the distances \ref{eq:metric} in the original dataset.

We subsequently perform clustering of this resulting data, using a Dirichlet Process Gaussian Mixture Model (DP-GMM) \citep{rasmussen1999infinite}, since the number of clusters is unknown (see previous section). The resulting algorithm for structuring map representations is shown in Figure \ref{fig:structalg}. It requires training data in the form of buildings with known representation structure in this form (acquired from recall lists) - see Chapter \ref{cha:structure}.

We point out that Equation \ref{eq:metric} constitutes a general framework for metric learning using any model capable of producing probability estimates that two instances belong together. This includes the entire family of generative models in machine learning (see e.g. \citep{bishop2006pattern}), as well as any discriminative model when combined with Platt scaling \citep{platt1999probabilistic} for transforming discrete outputs into probabilities. Two example applications of this general metric learning framework are semi-supervised clustering (extending the algorithm in Figure \ref{fig:structalg} by using semi-supervised GDA), or semi-supervised classification. See Appendix TODO for these examples and the evaluation of their performance with different constituent models.

\begin{figure}[h]
	\begin{pseudocode}{predictMapStructure}{X, knownX, knownStructure}
		1: corepresented \GETS \{\} \\
		2: notcorepresented \GETS \{\} \\
		3: \FOR i \in (1, |knownX|) \\
		4: \quad \FOR j \in (i+1, |knownX|) \\
		5: \quad \quad \IF knownStructure_i = knownStructure_j \\
		6: \quad \quad \quad corepresented \GETS corepresented \cup (knownX_i-knownX_j) \\
		7: \quad \quad \textbf{else} \\
		8: \quad \quad \quad notcorepresented \GETS notcorepresented \cup (knownX_i-knownX_j) \\
		9: \ \ \mu_{co} \GETS mean(corepresented) \\
		10: \Sigma_{co} \GETS cov(corepresented) \\
		11: coprior \GETS \frac{|corepresented|}{|knownX|} \\
		11: \mu_{not} \GETS mean(notcorepresented) \\
		12: \Sigma_{not} \GETS cov(notcorepresented) \\
		13: notprior \GETS \frac{|notcorepresented|}{|knownX|} \\
		14: D \in \mathbb{R}^{|X|x|X|} \\
		15: \FOR i \in (1, |X|) \\
		16: \quad \FOR j \in (i+1, |X|) \\
		17: \quad \quad D_{i,j} \GETS 1 - \frac{coprior \cdot \mathcal{N}((X_i-X_j); \mu_{co}, \Sigma_{co})}{coprior \cdot \mathcal{N}((X_i-X_j); \mu_{co}, \Sigma_{co}) + notprior \cdot \mathcal{N}((X_i-X_j); \mu_{not}, \Sigma_{not})} \\
		18: embedding \GETS MDS(D) \\
		19: structure \GETS DPGMM(embedding) \\
		20: return(structure)
	\end{pseudocode}
	\caption[Algorithm for predicting spatial representation structure]{\textbf{Algorithm for predicting participants' spatial representation structure}, given the features of the new buildings to be structured, and given buildings with known structure (from a previous experiment) specifying which of these buildings were co-represented.}
	\label{fig:structalg}
\end{figure}

%\section{Image processing, segmentation, and recognition}

%point out comp aspects WHEREVER POSSIBLE / APPROPRIATE 

%review of methods point out COMP./MATH

%point out what you can get / what you cant get, specifically saying limitations

%limitations 

%\section{Empirical methods}
%
%\subsection{Bayesian localization}
%
%The computational models presented below are based on the hypotheses summarized in Table \ref{tbl:hyp} in the Introduction. Previously gathered and published data was available for testing the hypotheses that hippocampal place cells might encode uncertainty and use it for approximate Bayesian inference, in the form of neural activity recorded using head-mounted intracranial electrodes in behaving rats in multiple environments, collected outside this PhD by \citep{burke2011influence, okeefe1996geometric, Odobescu2010} - see Chapter \ref{cha:bayespc}. Place cells also play a crucial role in human spatial memory \citep{ekstrom2003cellular}. But since the cognitive models below are concerned with human cognition, we additionally used human behaviour data published by \citep{nardini2008development} to evaluate the Bayesian localization model on a behavioural level. In this  see Chapter \ref{cha:lida}.
%
%\subsection{Bayesian nonparametric clustering}
%
%Although 
%
%\section{Computational methods} 

\chapter{Review of computational cognitive models of spatial memory}
\label{cha:nnreview}

\textbf{Publication 1 / 4.} Madl T., Chen K., Montaldi D. \& Trappl R., 2015. Computational cognitive models of spatial memory in navigation space: A review. \textit{Neural Networks, 65, 18-43.}

\newpage

\addtocounter{page}{-1}

\includepdf[pages={-}, 
addtolist={
	1, figure, {Grid cells, place cells, boundary-related cells, head-direction cells, and the neuronal basis of self-motion information (p. 21)}, fig:nnrev:gcpcbvc,
	1, figure, {Overview of symbolic models evaluated in real-world environments (p. 25)}, fig:nnrev:symrl,
	1, figure, {Overview of symbolic models evaluated in simulated environments (p. 28)}, fig:nnrev:symsim,
	1, figure, {Two navigation strategies (p. 29)}, fig:nnrev:navstrat,
	1, figure, {Overview of neural network models evaluated in real-world environments (p. 30)}, fig:nnrev:nnrl,
	1, figure, {Overview of neural network models evaluated in simulated environments 1 (p. 32)}, fig:nnrev:nnsim1,
	1, figure, {Overview of neural network models evaluated in simulated environments 2 (p. 32)}, fig:nnrev:nnsim2,
	1, figure, {Overview of cognitive architectures evaluated in simulations 1 (p. 35)}, fig:nnrev:cogarch1,
	1, figure, {Overview of cognitive architectures evaluated in simulations 2 (p. 35)}, fig:nnrev:cogarch2,
	1, table, {Characteristics of the reviewed models (pp. 38-39)}, tbl:nnrev:char
}]{papers/nnreview.pdf}

\addtocounter{page}{-25}

\chapter{Bayesian integration of information in hippocampal place cells}
\label{cha:bayespc}

\textbf{Publication 2 / 4.} Madl T., Franklin S., Chen K., Montaldi D. \& Trappl R., 2014. Bayesian Integration of Information in Hippocampal Place Cells. \textit{PLoS ONE 9(3), e89762}

\vspace{1cm}
 
\textit Note: the manuscript was originally published with incorrect figure ordering. The correct figure order was published as a correction (doi: 10.1371/journal.pone. 0136128), but PLOS has decided to maintain the old manuscript with incorrect ordering online. The reprint below contains the corrected figure order. No other changes have been made to the online version.

\newpage

\addtocounter{page}{-1}

\includepdf[pages={-}, 
addtolist={
	1, figure, {Place field sizes, and predicted uncertainty, on an empty rectangular track (p. 4)}, fig:bayespc:pfrect,
	1, figure, {Place field sizes, and predicted uncertainty, on a circular track with objects (p. 5)}, fig:bayespc:pfcirc,
	1, figure, {Predicted and recorded place fields in environment B (p. 6)}, fig:bayespc:pfb,
	1, figure, {Neuronal implementation of Bayesian inference based on coincidence detection (p. 8)}, fig:bayespc:cd,
	1, figure, {Density of place cell spikes, and predicted uncertainty, on a circular track with objects (p. 9)}, fig:bayespc:density,
	1, figure, {Place field sizes, and predicted uncertainty, on a circular track with objects, using the extended model (p. 10)}, fig:bayespc:pfextended,
	1, figure, {Errors of coincidence-based multiplication based on a simple integrate-and-fire model (p. 11)}, fig:nnrev:cderr
}]{papers/bayespc.pdf}

\addtocounter{page}{-15}

\chapter{The structure of spatial representations}
\label{cha:structure}

\textbf{Publication 3 / 4.} Madl T., Franklin S., Chen K., Trappl R. \& Montaldi D., submitted. Exploring the structure of spatial representations. \textit{Cognitive Processing}

\newpage

\addtocounter{page}{-1}

\includepdf[pages={-}, 
addtolist={
	1, figure, {Formalizing relative feature importances for grouping objects (p. 4)}, fig:mapstructure:formalizing,
	1, figure, {A part of the real-world memories experiment interface of Experiments 1 and 3 (p. 5)}, fig:mapstructure:rlscrn,
	1, figure, {A part of the virtual reality experiment interface of Experiment 2 (p. 6)}, fig:mapstructure:vrscrn,
	1, figure, {The recall sequence-based method used to extract cognitive map structure (p. 8)}, fig:mapstructure:recallsequence,
	1, figure, {Overview over the 149 cities in which participants' memory structures were inferred (p. 9)}, fig:mapstructure:map149,
	1, figure, {Correlations between probabilities of being on the same sub-map, and distances along each feature, (p. 16)}, fig:mapstructure:corr1,
	1, figure, {Variability of features influencing cognitive map structure (p. 17)}, fig:mapstructure:variability,
	1, figure, {The decision hyperplane method for inferring feature importances and generating environments (p. 19)}, fig:mapstructure:dechyp,
	1, figure, {Results of a predictive clustering model using subjects' feature importances, learned using the decision hyperplane approach (p. 22)}, fig:mapstructure:dechypresults,
	1, figure, {Learning subject-specific models for predicting cognitive map structure (p. 24)}, fig:mapstructure:gdaopt,
	1, figure, {Estimated maximum possible prediction rate using the data in Experiment 3 (p. 26)}, fig:mapstructure:maxpred,
	1, figure, {Accuracies obtained by predicting participant’s map structures using DP-GMM clustering under the learned subject-specific models (p. 29)}, fig:mapstructure:accuracies,
	1, figure, {Possible obstacles to predicting subject cognitive map structures (p. 33)}, fig:mapstructure:obstacles,
	1, table, {Effects of spatial representation structure on distance estimation, walking time estimation, and response times (p. 13)}, tbl:mapstructure:effects,
	1, table, {Prediction accuracies (and Rand indices) in Experiment 3 (p. 28)}, tbl:mapstructure:predacc
}]{papers/mapstructure.pdf}

\addtocounter{page}{-36}

\chapter{Towards real-world capable spatial memory in the LIDA cognitive architecture}
\label{cha:lida}

\textbf{Publication 4 / 4.} Madl T, Franklin S, Chen K, Montaldi D \& Trappl R, submitted. Towards real-world capable spatial memory in the LIDA cognitive architecture. \textit{Biologically Inspired Cognitive Architectures}

\newpage

\addtocounter{page}{-1}

\includepdf[pages={-}, 
addtolist={
	1, figure, {Spatially relevant brain areas and LIDA modules (p. 5)}, fig:lida:brain,
	1, figure, {Extensions to add spatial abilities to LIDA (p. 7)}, fig:lida:extensions,
	1, figure, {Representations in Extended PAM in a simulated environment (p. 8)}, fig:lida:epam,
	1, figure, {Approximate Bayesian cue integration in spiking neurons (p. 10)}, fig:lida:apprbayes,
	1, figure, {Route planning on recurrently interconnected place nodes (p. 10)}, fig:lida:routeplanning,
	1, figure, {Loop closing performed by the Map correction SBC (p. 12)}, fig:lida:loop,
	1, figure, {Position errors and standard deviations in the cue integration experiment by Nardini et al. (2008) (p. 14)}, fig:lida:nardini,
	1, figure, {Comparison with human and model errors over all environments (p. 15)}, fig:lida:comparison
}]{papers/lidaspm.pdf}

\addtocounter{page}{-17}

\chapter{Discussion}
\label{cha:discussion}

In Chapters \ref{cha:bayespc}-\ref{cha:lida} above, we have argued for the necessity of probabilistic mechanisms in spatial cognition when faced with a complex, uncertain environment perceived through noisy sensors. Although by no means conclusive, we have presented evidence that

\begin{enumerate}
	\item hippocampal place cells represent spatial uncertainty, 
	\item they can perform approximate Bayesian inference,
	\item the representations by recently active place cells can be corrected near-optimally when revisiting a place through reverse replay, and
	\item spatial representation structure arises from clustering under a metric defined over features including distance and visual and functional similarity.
\end{enumerate}

We have also integrated these suggested probabilistic mechanisms into LIDA, and embodied the resulting cognitive architecture in a robotic simulation. In this Chapter we first briefly outline our tentative neural model of these same mechanisms (Figure \ref{fig:neurimpl}). Subsequently, we discuss the abilities, shortcomings, and missing functionalities of our models, and their consistency with related empirical findings, from a cognitive science perspective. 

\nocite{deshmukh2013}

\begin{figure}[h]
	\centering
	\includegraphics[width=\textwidth]{img/neurimpl3}
	\caption[Probabilistic spatial localization and mapping on the neuronal level]{\textbf{Probabilistic spatial localization and mapping on the neuronal level}. A: Neural correlates of localization (see Chapter \ref{cha:nnreview} for details; and (Deshmukh et al., 2013) for evidence of landmark vector cells). B: Probabilistic graphical model of the simultaneous localization and mapping problem \citep{thrun2008simultaneous}. Instead of capturing all correlations introduced through the landmarks, which requires vast computational resources, our model separately solves Bayesian localization with only local landmarks, and map correction (`pose optimization' in SLAM) with only loop closure constraints. See Chapter \ref{cha:methods} for notation and details. C: Illustration of firing fields during localization. Coloured dots represent spikes of the respective cells at specific locations. Path integration (grid cells) and boundary and landmark information (border cells, landmark vector cells) is integrated in place cells, using coincidence detection (rejection sampling) to obtain a near-optimal location estimate. This new estimate is used to update grid cell representations via phase reset to combat accumulating path integration errors (see Chapter \ref{cha:bayespc}). D: Illustration of a small loop (firing fields 1-6) which can be corrected upon recognizing the same landmark at positions 1 and 6 via reverse replay, by reactivating place cells 6-1 and shifting their place fields proportionally (see Chapter \ref{cha:lida}).} 
	\label{fig:neurimpl}
\end{figure}

%\section{A neural model of Bayesian mechanisms in spatial cognition}

\section{Other mechanisms and representations involved in spatial navigation}

%Naturally, explaining some amount of variance in a handful of datasets is not sufficient to validate a cognitive model. Ideally, a broad range of behaviour data from different tasks should be explainable using the same model with the same parameters, which is the ultimate goal of the LIDA cognitive architecture \citep{franklin2013lida} - but such extensive validation is beyond the scope of the current work. In this work, we have merely aimed to maximize consistency with established findings related to spatial cognition, implementability within known neural structures, and fit empirical data. The result chapters above have mostly focused on the latter. In this Chapter, we will briefly discuss the former two.

Tables \ref{tbl:spmech} and \ref{tbl:sprep} summarize the processes and representations involved in spatial navigation in biological cognition. The first columns provide overviews of these mechanisms and representations, based on Figure 1 in \citep{wolbers2010determines}. The second column indicates the corresponding mechanism in our final LIDA-based model, as described in Chapter \ref{cha:lida}. The rightmost column highlights some major elements missing from the models presented here but required for spatial navigation. 

%\begin{tabular}[c]{@{}c@{}} a \\ b \end{tabular}

\begin{table*}[h]
	\centering
	{ %\renewcommand{\arraystretch}{1.2}
		\begin{tabu}{c|c|c}
			$\downarrow$ {\textbf{Mechanism}} & {\textbf{In our model}} & {\textbf{Not implemented}} \\ \tabucline[3pt]{-}
			\multicolumn{3}{c}{\textbf{Spatial computations}} \\ \hline
			Space perception & \begin{tabular}[c]{@{}c@{}} Limited (depth from \\ stereo disparity*) \end{tabular} & \begin{tabular}[c]{@{}c@{}} Estimating size, shape, \\ movement, orientation, ... \end{tabular} \\ \hline
			Self-motion perception & \begin{tabular}[c]{@{}c@{}} Surrogate: odometry* \end{tabular} & \begin{tabular}[c]{@{}c@{}} Motor efference, proprio- \\ ceptive \& vestibular senses \end{tabular} \\ \hline
			\begin{tabular}[c]{@{}c@{}} Translation btw. ego- and \\ allocentric reference frames \end{tabular} & \begin{tabular}[c]{@{}c@{}} Limited: Perspective \\projection via \\homography* \end{tabular} & \begin{tabular}[c]{@{}c@{}} Plausible translation \\ mechanism \end{tabular} \\ \hline
			\begin{tabular}[c]{@{}c@{}} Computing directions and \\ distances to unseen goals \end{tabular} & \begin{tabular}[c]{@{}c@{}} Route plan SBC \\ (following gradient \\ on a hierarchical grid) \end{tabular} & \begin{tabular}[c]{@{}c@{}} Explicit direction \\ estimation, systematic \\ errors in estimation \end{tabular} \\ \hline
			\begin{tabular}[c]{@{}c@{}} Imagining shifts in \\ spatial perspective \end{tabular} & - & Sensory imagery \\ \hline
			
			\multicolumn{3}{c}{\textbf{Executive processes}} \\ \hline
			Novelty detection & - & \begin{tabular}[c]{@{}c@{}} Perceptual recognition of \\ known or novel places \end{tabular} \\ \hline
			\begin{tabular}[c]{@{}c@{}} Selection and maintenance \\ of navigational goals \end{tabular} & \begin{tabular}[c]{@{}c@{}} Attention codelets* \\ \& global broadcast* in \\ LIDA's cognitive cycle \end{tabular} & \begin{tabular}[c]{@{}c@{}} Reward representations, \\ reinforcement learning \end{tabular} \\ \hline
			Route planning or selection & \begin{tabular}[c]{@{}c@{}} Route plan SBC \\ (following gradient \\ on a hierarchical grid) \end{tabular} & \begin{tabular}[c]{@{}c@{}} Expectation violation / \\ confirmation monitoring, \\ re-planning, homing...\end{tabular} \\ \hline
			\begin{tabular}[c]{@{}c@{}} Uncertainty/Conflict \\ resolution \end{tabular} & \begin{tabular}[c]{@{}c@{}} Partial: Bayesian \\ integration \end{tabular} & \begin{tabular}[c]{@{}c@{}} Conflicting cues, \\ cues other than odometry \\ \& estimated distance  \end{tabular} \\ \hline
			Resetting mechanisms & \begin{tabular}[c]{@{}c@{}} Partial: maximum \\ likelihood correction \end{tabular} & \begin{tabular}[c]{@{}c@{}} Kidnapped robot \\ problem \end{tabular} \\ \hline
		\end{tabu}
	}
	\caption[Cognitive mechanisms involved in spatial navigation]{\textbf{Cognitive mechanisms involved in spatial navigation}, based on \citep{wolbers2010determines}. *: an ability of our model making use of existing implementations (in the LIDA cognitive architecture or the Robot Operating System).}
	\label{tbl:spmech}
\end{table*}

\clearpage

\begin{table*}[h]
	\centering
	{ %\renewcommand{\arraystretch}{1.2}
		\begin{tabu}{c|c|c}
			$\downarrow$ {\textbf{Representation}} & {\textbf{In our model}} & {\textbf{Not implemented}} \\ \tabucline[3pt]{-}
			\multicolumn{3}{c}{\textbf{Online representations}} \\ \hline
			Self-position and orientation & `Self' PAM node & - \\ \hline
			\begin{tabular}[c]{@{}c@{}} Egocentric self-to-object \\ directions and distances \end{tabular} & \begin{tabular}[c]{@{}c@{}} Limited (depth from \\ stereo disparity*) \end{tabular} & \begin{tabular}[c]{@{}c@{}} Egocentric vectors (e.g. \\ `reach vectors' in area 5a) \end{tabular} \\ \hline
			\begin{tabular}[c]{@{}c@{}} Allocentric object-to-object \\ directions and distances \end{tabular} & \begin{tabular}[c]{@{}c@{}} Indirect (on map \\ representation, but not \\ perceptually) \end{tabular} & \begin{tabular}[c]{@{}c@{}} Allocentric visuo- \\ spatial representations \end{tabular} \\\hline
			Route progression & `Route' PAM nodes & Expectations \\\hline
			Navigation goals & `Goal' PAM nodes & Rewards \\\hline
			\multicolumn{3}{c}{\textbf{Offline representations}} \\ \hline
			\begin{tabular}[c]{@{}c@{}} Memories of local \\ views and places \end{tabular} & \begin{tabular}[c]{@{}c@{}} Partial (in pre-conscious \\ working memory, not \\yet in long-term memory) \end{tabular} & \begin{tabular}[c]{@{}c@{}} Long-term memory \\ representations \end{tabular} \\\hline
			\begin{tabular}[c]{@{}c@{}} Enduring, hierarchical \\ representations of an \\ environment (ego-/allocentric) \end{tabular} & \begin{tabular}[c]{@{}c@{}} Hierarchical maps \\ consisting of \\ `place nodes' \end{tabular} & \begin{tabular}[c]{@{}c@{}} Hierarchical egocentric \\ representations \end{tabular} \\\hline
			Networks of habitual routes & \begin{tabular}[c]{@{}c@{}} Context-action-result chains \\ in Procedural Memory* \end{tabular} & - \\\hline
		\end{tabu}
	}
	\caption[Representations involved in spatial navigation]{\textbf{Representations involved in spatial navigation}, based on \citep{wolbers2010determines}}
	\label{tbl:sprep}
\end{table*}


\section{Limitations and shortcomings}

In addition to mechanisms and representations playing an important role in spatial navigation but not yet implemented in our model (Tables \ref{tbl:spmech} and \ref{tbl:sprep}), there are several shortcomings of our models, which we outline in this Section. They can roughly be grouped into three categories: computational shortcomings, psychological implausibilities, and neural implausibilities.

\subsection{Computational shortcomings}

We have pointed out in Chapters \ref{cha:intro} and \ref{cha:methods} that the goal of this work was not to optimize for performance (but rather computational cognitive modelling), and that these problems can be solved more optimally and accurately, given enough computational resources. Accuracy and performance of spatial representations are the goals of Simultaneous Localization and Mapping (SLAM) in mobile robotics \citep{thrun2008simultaneous}. 

State of the art solutions to the SLAM problem can infer robot and landmark locations down to a few centimetres accuracy or better, but usually require $5-25 \%$ of the processing power of a current Intel Core i7-3630QM CPU to do so \citep{machado2013evaluation}, even when just mapping a small room, which amounts to $4-20$ billion floating point operations per second\footnote{Based on Intel i7 specifications, retrieved from  \url{http://download.intel.com/support/processors/corei7/sb/core_i7-3600_m.pdf}}. Achieving the same in large-scale outdoor environments would require even more computational resources.

Figure \ref{fig:endtoendslam} shows the structure of modern end-to-end SLAM systems \citep{wang2015}, such as e.g. \citep{newman2011describing}. Components depending on the specific sensors and actuators ('front-end') are usually separated from the sensor-independent optimization part (`back-end'). In our final model described in Chapter \ref{cha:lida}, the `front-end' roughly corresponds to the functionality of the Bayesian localization SBC, and the `back-end' to that of the Map correction SBC. Both functionally correspond to hippocampal place cells, with the former mechanism implemented by coincidence detection and the latter through reverse replay. 

\begin{figure}[h]
	\centering
	\includegraphics[width=\textwidth]{img/endtoendslam}
	\caption[Components of a modern end-to-end SLAM system]{\textbf{Components of a modern end-to-end SLAM system}. From \citep{wang2015}} 
	\label{fig:endtoendslam}
\end{figure}

The two main computational shortcomings compared to modern SLAM include 1) not explicitly modelling rotations (thus avoiding non-linearity caused by robots which can turn), and 2) not explicitly optimizing landmark constraints (only path integration and loop closure constraints). These cause inferior localization and mapping accuracy compared to modern SLAM. However, they have allowed us to map Bayesian mechanisms to well-known neural correlates and mechanisms, and to implement simple models successfully replicating behaviour data, while still retaining the ability to tackle the uncertainty and noise problem in a realistic robotic simulation.

Although brains may well be capable of the processing power required by a SLAM system, it is unlikely that they work the way modern SLAM solutions do (performing thousands of linear algebra operations serially) \citep{thrun2008simultaneous}. Furthermore, human long-term memories are far from being as accurate as these SLAM systems, as shown e.g. in Figures \ref{fig:lida:nardini} and \ref{fig:lida:comparison} in Chapter \ref{cha:lida} or by research regarding sketch maps, e.g. \citep{rovine1989sketch, wang2009accuracy}. Nevertheless, there is value in looking on information processing in brains through the lens of normative models, of mathematical formulations of the problem to be solved; and of their implementability in brains and minds. 

\subsection{Psychological implausibilities}

Apart from implementation details (in brains and in LIDA), on Marr's (1976) algorithmic level, three mechanisms were suggested in this thesis: 1) a cue integration mechanism for localization, 2) correction of cognitive maps when re-visiting places, and 3) cognitive map structuring through clustering. Despite of their ability to fit behaviour data as described in Chapters \ref{cha:bayespc}-\ref{cha:lida}, there are some psychological findings which are inconsistent with these mechanisms. 

First, our models have focused on adult cognition, and have ignored developmental findings. Visual spatial integration progressively improves in children between 5 and 14 years of age \citep{kovacs1999late}. Spatial cue integration, while close to the Bayesian optimum in adults, seems to require a long developmental process; and children do not seem to integrate spatial cues, instead switching between exclusively using path integration or landmark information from trial to trial \citep{nardini2008development}. It is difficult to model this behaviour in our Bayesian framework.

In terms of adult spatial cognition, there are shortcomings in how landmarks are recognized. In the current model, any objects recognized by the CNN briefly described in \ref{cha:lida} constitutes a landmark. However, in human (and animal) cognition, landmarks have to be reliable, salient, stable (unmoving), and possibly distal \citep{lew2011looking}. These criteria defining landmarks for biological spatial cognition are not accounted for in the model. Neither are cues in the form of landmark arrays (e.g. humans use the natural axes of regular arrays of object as a reference frame) \citep{lew2011looking, burgess2006spatial}. 

Phenomena observed in environments with competing cues (e.g. landmarks), where the information from the cues is not integrated, are also difficult to model in our probabilistic framework. Examples include `overshadowing' (where the effect of a cue on an animal's behaviour may be reduced or eliminated when another, more salient cue is introduced) and `blocking' (where a second cue is added after an animal has been trained with the first, but the animal cannot use the second cue without the first) \citep{chamizo2003acquisition}. Some evidence of landmark overshadowing and blocking in humans exists, e.g. \citep{spetch1995overshadowing, prados2011blocking}, and it has been argued that unlike the role of boundaries, associative reinforcement (and not a map-like representation) may be a better explanation for landmark learning \citep{doeller2008distinct}. 

Navigation based on two complementary systems running in parallel (a cognitive mapping system using the described mechanisms, and a reward-based associative learning system based on LIDA's procedural memory) is conceptually consistent with blocking and overshadowing, and may be able to explain these findings. We have not implemented this computationally, however; and the extent of cooperation / competition between these systems is not yet clear, even on a theoretical level \citep{lew2011looking, cheng201325}.

In addition to the role of landmarks, a `geometric module' for navigation has been proposed, originally to explain errors which would have been avoidable if perceptual as opposed to geometric cues had been used (such as rats learning there is food in the corner of a rectangular environment, but often searching in the diagonally opposite corner of the environment, which was geometrically - but not perceptually - equivalent) \citep{cheng1986purely}. Similar geometry-based behaviour has been observed in young children, e.g. by \cite{huttenlocher1999spatial} (see also \citep{cheng201325}). Recent findings cast the existence of a dedicated geometric module for orientation and navigation in doubt \citep{cheng2008whither}. Nevertheless, empirical observations of such errors (which are consistent with geometry-based orientation, but could be avoided by making use of perceptual features/landmarks) are inconsistent with our model, which does not make such errors.

Other types of systematic errors in spatial representations have been pointed out in the literature which our model does not account for in its current form. Distortions result from the hierarchical organization in cognitive maps \citep{tversky1992distortions,hirtle1985evidence} - which, however, could easily be incorporated into the model, given that it already learns these hierarchies (all that is required is implementing an error function/mechanism). However, there are also systematic distortions of spatial representations which are not easily accounted for in this framework. They include effects of perspective (where participants are asked to imagine themselves when asked to estimate spatial relations), of cognitive reference points (distance judgements made from landmark A to building B usually differ from those made from building B to landmark A), and of detours or barriers (the length of circuitous routes is usually overestimated) - see \citep{tversky1992distortions, tversky2003navigating}. Differences in viewpoints used when learning spatial representations and when having to use them also cause systematic errors (e.g. \citep{shelton2001systems, Shelton2004, burgess2006spatial}) which have been neglected by the current models.

Finally, the current model, when forced to explore very large regions without being allowed to ever revisit known places, can incur catastrophically large errors to its learned representations, making the learned map largely useless (we know of no such effect observed in humans). It is likely that in very large scale environments, humans make use of several parallel mechanisms including spatial reasoning, as well as of prior knowledge of the structure of the environment (e.g. the usual shapes of roads), none of which have been included in the model. 

We note that to our knowledge, no current computational cognitive model of spatial memory achieves full consistency with every empirical finding, while being capable of running in realistic environments at the same time (see review in Chapter \ref{cha:nnreview}). We have argued that our approach is a step in the direction of such models, which can be the case even if it does not support modelling some known aspects of spatial cognition. As long as the basic premises hold (that brains can represent uncertainty, and can perform approximate Bayesian inference), and if the shortcomings can be corrected in future models in a cognitively plausible fashion, the probabilistic approach to spatial cognition remains viable.


\subsection{Neural implausibilities}

In terms of consistency with neuroscientific findings, we have to distinguish between the final computational cognitive model based on the LIDA cognitive architecture (Chapter \ref{cha:lida}), and between the neural hypotheses / suggested neural mechanisms regarding uncertainty representation and error correction in the hippocampus. We omit discussing the neural plausibility of the map structuring model introduced in Chapter \ref{cha:nnreview}, since we have not described any neural implementation of this mechanism, and only validated it behaviourally (but see e.g. \citep{shi2009neural} or \citep{sanborn2015types} for possible neural implementations of hierarchical Bayesian models, to which the DP-GMM belongs). It is to our knowledge the first model able to predict spatial representation structure on the individual level; and developing a biologically plausible implementation in addition to a normative and algorithmic model would have exceeded the time available for this PhD.

Regarding the final model (Chapter \ref{cha:lida}), LIDA intends to be a model of minds, not brains (it is a model on Marr's algorithmic and not implementation level) - see \citep{franklin2012global, franklin2013lida} for discussions of the relationship between LIDA and the underlying neuroscience. Nevertheless, we briefly point out a few mechanisms of the model in Chapter \ref{cha:lida} (LIDA extended by the described probabilistic spatial mechanisms and embodied on a robot) which do not directly correspond to known neural processes. 

The first salient difference is the visual recognition system, for which we used existing implementations to make this work tractable within one PhD. Specifically, we used convolutional neural networks for recognizing objects \citep{szegedy2014going} and roads \citep{brust2015convolutional}, which have been designed for maximizing recognition performance, not for neural plausibility. Curiously, they do seem to learn representations that are very similar to those recorded in human and primate inferior temporal cortex \citep{khaligh2014deep, yamins2013hierarchical}. But their conventional training algorithms are not implementable in biological neurons \citep{stork1989backpropagation, bengio2015towards}. Developing a plausible recognition system would have exceeded the scope of this PhD. The same is true for motor control, for which we used existing drivers of the Robot Operating System\footnote{See \url{http://wiki.ros.org/pid} and \url{http://gazebosim.org/tutorials?tut=drcsim_fakewalking&cat=drcsim}} (which are by no means brain-like).

In terms of the spatial extensions to LIDA, the biggest discrepancy is the regular grid formed by the `place nodes' (Chapter \ref{cha:lida}). Place cells do not seem to map the surface of an environment in any systematic fashion \citep{o1998place}. It would be more accurate to think of `place nodes' as combining several underlying spatially relevant cell types, including entorhinal grid cells, which do form regular grids (although triangular and not rectangular) \citep{moser2008place}. Grid cells also facilitate estimating directions and distances \citep{bush2015using}. However, the simple route planning strategy (based on spreading activation on hierarchical grids of place nodes) is not a faithful model of navigation in the hippocampal entorhinal complex, as it heavily relies on a regular structure and on specific link weights depending on distances and obstacles. \cite{bush2015using} reviews four more biologically plausible network models on Marr's implementation level. However, LIDA is concerned with the algorithmic level - and there is published behavioural evidence for such a mechanism \citep{mueller2013pathfinding}. We have also succeeded in replicating several multi-goal route planning datasets using our simple model (see Appendix \ref{apx:lidaspm}), which substantiates its cognitive plausibility. 

On the other hand, the plausibility of the probabilistic framework for cognitive modelling does require, at the very least, the possibility of neurally implementing Bayesian inference. To show evidence of this possibility, we have compared the firing of hippocampal place cells to predictions of a Bayesian model, and have suggested they might be able to represent uncertainty and perform approximately optimal inference (see Chapter \ref{cha:bayespc}). These are hypotheses on the neuronal level. As such, they can be compared to neuroscientific findings - and they do seem to be inconsistent with some, as summarized below.

First, humans with hippocampal lesions, although spatially impaired, do seem to be able of spatial navigation. For example, \citep{teng1999} report a patient with damaged medial temporal areas who was able to describe routes, detours, and directions between landmarks in an environment he has learned early, before the damage. The authors suggest that the role of the hippocampus is time-limited, mostly concerning consolidation, and that long-term spatial memories are available after consolidation even with a lesioned hippocampus. Similar observations of largely unimpaired topographical abilities in patients with hippocampal damage were found by \citep{rosenbaum2000, rosenbaum2005}; although these patients did show some types of impairments (few recalled landmarks on sketch maps, no detailed geographical knowledge, impaired landmark recognition). A later study by \cite{maguire2006} reinforced the implication that although accessing long-established spatial memories is still possible with a damaged hippocampus, topographical knowledge of landmarks and of the relationships between them is impaired. Naturally, the ability to learn new spatial representations is also heavily impaired. Nevertheless, some functionalities requiring allocentric representations seem to be available to patients with hippocampal lesions, which is problematic for the `cognitive map' hypothesis in general, as well as for our model.

Second, the firing fields of place cells do not behave like unique, one-to-one representations of location. Some place cells (a minority) have more than one firing field \citep{burke2011influence}, and there do not seem to be systematic commonalities between these (e.g. similar distances to surroundings) as would be predicted by a model using these firing fields as probability distributions. Place fields are also not always regular and elliptic, as prescribed by the simplest Gaussian model in Chapter \ref{cha:bayespc} (although this is not an issue for the particle filter-based formulation in Chapter \ref{cha:lida}, which can represent multimodal distributions). Furthermore, it is not the case that place fields close to boundaries are always smaller than those further away, as would be predicted if they solely represented uncertainty. For example, firing fields of cells in dorsal hippocampus are generally smaller than those of cells in more ventral areas \citep{kjelstrup2008finite}. Finally, there are several phenomena observed in recordings from place cells of behaving animals which do not easily fit into a probabilistic model. These include remapping \citep{colgin2008understanding} and theta phase precession \citep{skaggs1996theta}.

However, these inconsistencies do not falsify the possibility of an approximate Bayesian inference mechanism operating in the hippocampus in parallel to several other mechanisms not accounted for (and in some cases inconsistent with) such a mechanism. Brains exhibit a high degree of redundancy, and there is no reason to assume that one cell type only performs one function. Over-reliance on only a single or few place cells inconsistent with the statistical optimum could destroy the models functionality. However, larger ensemble of place cells, a majority of which do represent location estimates and their associated approximate uncertainty, can still facilitate approximately optimal localization if the contradicting information in the ensemble (representing other things, such as an episodic memories \citep{tulving1998episodic}) is a minority. The approximate Bayesian hippocampus hypothesis could be falsified if the number of place cells used for localization and having firing fields inconsistent with Bayesian uncertainty predictions could be shown to be a majority. However, this does not seem to be the case in the recordings and environments investigated here (see Chapter \ref{cha:bayespc}).

We can further support the claim of multiple parallel hippocampal mechanisms, one of which may be approximate Bayesian inference, using three observations. First, the reasonably good fit of Bayesian predictions with empirical place field sizes reported in Chapter \ref{cha:bayespc} would be extremely unlikely to occur by chance, given that hundreds of place fields were included in the comparison. Second, our particle filter localization model is largely robust to artificially increasing or decreasing the variance of the samples at some places\footnote{In fact, adding random samples, independently from the Bayesian prediction, was one of the early methods used in robotics to combat `particle depletion' and to increase the chances of the robot being able to recover its correct location in particle filter-based SLAM \citep{thrun2005probabilistic}.}, which is a rudimentary way of simulating some place fields having a different size than prescribed by a Bayesian model. Finally, the uncertainties predicted by a sampling-based localization model can also successfully explain the frequency distribution of place field sizes (see comparison in Appendix \ref{apx:pfev}). The small amount of deviation between the frequency distribution of uncertainties in the model, and between the frequency distribution of place field sizes, suggests that place fields inconsistent with Bayesian uncertainty were a small minority in this environment. 

%\subsection{Neural implausibilities of approximately Bayesian place cells}

% # PF

% REMAPPING

% getting Sigma is slow

% map correction - explicit distance

% hc power / activity ~ mvmnt speed

%\subsection{Neural implausibilities of map correction by reverse replay}

% map correction - explicit distance


%\section{Alternative explanations}

% Gigerenzer line of work, heuristics instead of approximate optimality - see eg Sanborn 2015

% Barrera, RatSLAM


% what does all this mean? interpret, evaluate

% strengths
% limitations



% neural implementations

% point out that evidence for GC updating has since been found

% caveats, problems, limitations

%Is there agreement or disagreement with previous work?
	% have to properly discuss BVC model (see Hartley slides)
	
%Interpret results in terms of background laid out in the introduction - what is the relationship of the present results to the original question?
%What is the implication of the present results for other unanswered questions

% There are usually several possible explanations for results. Be careful to consider all of these rather than simply pushing your favorite one

%What are the things we now know or understand that we didn't know or understand before the present work?

% What is the significance of the present results: why should we care? 

% This section should be rich in references to similar work and background needed to interpret results.


%-------

% examiners are looking for (In Discussion AND conclusions:)
%- Is the candidate aware of possible limits to confidence/ reliability/validity of the work?
%- Have the main points to emerge from the results been picked up for discussion?
%- Are there links made to the literature?
%- Is there evidence of attempts at theory building or reconceptualisation of problems?Are there speculations?/re they well grounded in the results?

\chapter{Conclusion}
\label{cha:conclusion}

Humans live and act in a world they can only partially observe through imperfect sensors and process with an inherently noisy information processing system. In mathematics, probability theory has provided a framework for the representation and manipulation of uncertainty \citep{jaynes1996probability}. In this thesis, we have argued for the necessity of such a framework within the field of computational cognitive modelling as well. We have modelled and interpreted neuroscientific evidence in a probabilistic framework, providing one of the first examples of Bayesian inference on a single-neuron level, in order to provide the foundation of this argument (Chapter \ref{cha:bayespc}). 

Simply using existing algorithmic solutions of probabilistic localization, mapping, and clustering does not yield viable models of cognition, since these differ from biological cognitive processes in behaviour, computational requirements, and available information. However, most existing cognitive models of spatial memory, while plausibly modelling cognition, are unable to deal with sensory noise and uncertainty. We have provided a detailed review and comparison of such models in Chapter \ref{cha:nnreview}, and have suggested the ability to function in realistic environments as one of the main gaps in the literature.

In order to take a first step towards filling this gap, we have proposed computational models on Marr's (1976) algorithmic level of

\begin{itemize}
	\item self-localization (`\textit{where am I?}'),
	\item object localization (`\textit{where is this object?}'),
	\item map correction after revisiting a place (`\textit{I've been here before - now how do I fix my map?}'), 
	\item multi-goal route planning (`\textit{how do I get to these places?}'), and
	\item map structuring (`\textit{which map does this object belong to?}'),
\end{itemize}

Although these problems, with the exception of the last, are well-known in robotics, we have provided the - to our knowledge - first computational cognitive models which 1) are implementable in brains, 2) can reproduce behaviour data, 3) are part of a cognitive architecture, integrated with other cognitive processes, and 4) are able to function in realistic environments with noise and uncertainty (in a robotic simulation providing the exact same interfaces as a real robot \citep{rusu2007extending}). 

We have also shown, for the first time since the discovery of hierarchical structure in human spatial representations \citep{hirtle1985evidence}, that such structures are predictable based on geospatial, perceptual, and functional properties of the environment. We have provided evidence that Bayesian nonparametric clustering under a subject-specific distance metric accounts for a large majority of buildings belonging together in participants' established spatial memories.

Our models extend the `Bayesian brain' \citep{knill2004bayesian} and `Bayesian cognition' \citep{chater2010bayesian} paradigms one step towards navigation-space cognitive representations and processes. We hope they will encourage further research on coping with the challenges posed by the real world in computational cognitive models of spatial memory.



%What is the strongest and most important statement that you can make from your observations? 

%If you met the reader at a meeting six months from now, what do you want them to remember about your paper?
 
%Refer back to problem posed, and describe the conclusions that you reached from carrying out this investigation, summarize new observations, new interpretations, and new insights that have resulted from the present work.

%Include the broader implications of your results. 


%\section{Future Work}
% TODO


\bibliography{refs}    % this causes the references to be listed

\bibliographystyle{model5-names}
%% the bibliography style determines the format  in which both citations and references are printed,
%% other possible values are plain and abbrv
%%
%% If you want more control of the format of your citations you might want to take a look at
%% natbib.sty, which should be part of any standard LaTeX installation
%%
%% University regulations simply require that your citation style be consistent, so see what style
%% your supervisor recommends.

% Appendices start here
\appendix
\include{appendix1}
\end{document}
