\abstracttitle


%Abstract
%
%A good abstract explains in one line why the paper is important. It then goes on to give a summary of your major results, preferably couched in numbers with error limits. The final sentences explain the major implications of your work. A good abstract is concise, readable, and quantitative. 
%Length should be ~ 1-2 paragraphs, approx. 400 words. 
%Absrtracts generally do not have citations.
%Information in title should not be repeated. 
%Be explicit. 
%Use numbers where appropriate.
%Answers to these questions should be found in the abstract: 
%What did you do? 
%Why did you do it? What question were you trying to answer? 
%How did you do it? State methods.
%What did you learn? State major results. 
%Why does it matter? Point out at least one significant implication.

% Single spacing can be turned on for the abstract
%
{
\singlespacing
\setstretch{1.0}

%However, since brains have been shaped by the challenges of the physical world, cognitive models should take them into account.

Existing computational cognitive models of spatial memory often neglect difficulties posed by the real world, such as sensory noise, uncertainty, and high spatial complexity. On the other hand, robotics is unconcerned with understanding biological cognition. This thesis takes an interdisciplinary approach towards developing cognitively plausible spatial memory models able to function in realistic environments, despite sensory noise and spatial complexity. 

We hypothesized that Bayesian localization and error correction accounts for how brains might maintain accurate location estimates, despite sensory errors. We argued that these mechanisms are psychologically plausible (producing human-like behaviour) as well as neurally plausible (implementable in brains). To support our hypotheses, we reported modelling results of neural recordings from rats (acquired outside this PhD), constituting the first evidence for Bayesian inference in neurons representing spatial location, as well as modelling human behaviour data. 

%We hypothesized that neurons involved in spatial representation, called hippocampal place cells, might perform approximate Bayesian localization and error correction. We developed computational models implementing these mechanisms, which we argued to be psychologically plausible (producing human-like behaviour) as well as neurally plausible (implementable in brains). To support our hypotheses, we reported modelling results of neural recordings from rats (acquired outside this PhD), constituting the first evidence for Bayesian inference in place cells, as well as modelling human behaviour data. 

%  collected in experiments performed online
%We also collected and modelled sketch map accuracy data in experiments performed online, substantiating the suggested map correction mechanism.

%In addition to dealing with noise and uncertainty, in realistic environments, spatial representations also have to be stored and used efficiently. Hierarchical representations help dealing with large amounts of spatial information by facilitating efficient retrieval search and route planning. It has been suggested that cognitive maps in humans are hierarchical, but the computational principles underlying these hierarchies have remained unknown.

%collected in online experiments

In addition to dealing with uncertainty, spatial representations have to be stored and used efficiently in realistic environments, by using structured representations such as hierarchies (which facilitate efficient retrieval and route planning). Evidence suggests that human spatial memories are structured hierarchically, but the process responsible for these structures has not been known. We investigated features influencing them using data from experiments in real-world and virtual reality environments, and proposed a computational model able to predict them in advance (based on clustering in psychological space).
%We validated our proposed mechanism using spatial memories of human subjects in over a hundred cities world-wide, and implemented a computational model able to predict, in advance, their sub-map structures based on our hypothesis.

%\vspace{0.8em}

%; integrating them with the other cognitive phenomena accounted for by LIDA

We have extended a general cognitive architecture, LIDA (Learning Intelligent Distribution Agent), by these probabilistic models of how brains might estimate, correct, and structure representations of spatial locations. We demonstrated the ability of the resulting model to deal with the challenges of realistic environments by running it in high-fidelity robotic simulations, modelled after participants' actual cities. Our results show that the model can deal with noise, uncertainty and complexity, and that it can reproduce the spatial accuracies of human participants. 
%Our LIDA-based cognitive software agent could reproduce the spatial errors of human participants in different recreated environments, substantiating the plausibility of the suggested models.

}
