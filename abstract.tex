\abstracttitle


%Abstract
%
%A good abstract explains in one line why the paper is important. It then goes on to give a summary of your major results, preferably couched in numbers with error limits. The final sentences explain the major implications of your work. A good abstract is concise, readable, and quantitative. 
%Length should be ~ 1-2 paragraphs, approx. 400 words. 
%Absrtracts generally do not have citations.
%Information in title should not be repeated. 
%Be explicit. 
%Use numbers where appropriate.
%Answers to these questions should be found in the abstract: 
%What did you do? 
%Why did you do it? What question were you trying to answer? 
%How did you do it? State methods.
%What did you learn? State major results. 
%Why does it matter? Point out at least one significant implication.

% Single spacing can be turned on for the abstract
%
{
\singlespacing
\setstretch{1.0}
Computational cognitive models of spatial memory often neglect difficulties posed by the real world, such as sensory noise, uncertainty, and high spatial complexity. However, since cognition and its neural bases have been shaped by the structure and challenges of the physical world, cognitive models should take them into account.

This work takes an interdisciplinary approach towards developing a cognitively plausible spatial memory model able to function in realistic environments, despite sensory noise and spatial complexity. We investigated how spatially relevant brain areas might maintain accurate location estimates, despite accumulating sensory noise, hypothesizing that hippocampal place cells might perform Bayesian localization, and that hippocampal reverse replay might play a role in correcting learned maps after revisiting known places. We developed computational models implementing these probabilistic mechanisms, which we argued to be psychologically plausible (producing human-like behaviour) as well as neurally plausible (implementable in brains). In support of the hippocampal Bayesian localization hypothesis, we reported modelling results of single-neuron recordings from rats (acquired outside this PhD), constituting the first evidence for Bayesian inference in place cells, as well as modelling behaviour data from humans. We also collected and modelled sketch map accuracy data in experiments performed online, substantiating the suggested map correction mechanism.

In addition to dealing with noise and uncertainty, in realistic environments, large-scale representations also have to be stored and used efficiently. Hierarchical representations help dealing with large amounts of spatial information by facilitating rapid and efficient retrieval search and route planning. It has been suggested that cognitive maps in humans are hierarchical, but the computational principles underlying these hierarchies have remained unknown. We investigated features influencing cognitive map structure using spatial memory data concerning real-world and virtual reality environments collected in online experiments, and proposed a computational mechanism (clustering in psychological space) which might give rise to sub-map structures, showing that it can predict these structures in participants' spatial memory in advance.
%We validated our proposed mechanism using spatial memories of human subjects in over a hundred cities world-wide, and implemented a computational model able to predict, in advance, their sub-map structures based on our hypothesis.

%\vspace{0.8em}

We have extended a general cognitive architecture (the LIDA model of cognition) by these Bayesian mechanisms for localization and map learning, correction, and structuring; integrating them with the other cognitive phenomena accounted for by LIDA. We demonstrated the ability of the resulting model to deal with the challenges of realistic environments by running it in high-fidelity robotic simulations, modelled after participants' actual cities, showing that it can deal with noise, uncertainty and complexity, and that it can reproduce the spatial accuracies of human participants. 
%Our LIDA-based cognitive software agent could reproduce the spatial errors of human participants in different recreated environments, substantiating the plausibility of the suggested models.

}
