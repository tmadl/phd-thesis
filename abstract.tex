\abstracttitle


%Abstract
%
%A good abstract explains in one line why the paper is important. It then goes on to give a summary of your major results, preferably couched in numbers with error limits. The final sentences explain the major implications of your work. A good abstract is concise, readable, and quantitative. 
%Length should be ~ 1-2 paragraphs, approx. 400 words. 
%Absrtracts generally do not have citations.
%Information in title should not be repeated. 
%Be explicit. 
%Use numbers where appropriate.
%Answers to these questions should be found in the abstract: 
%What did you do? 
%Why did you do it? What question were you trying to answer? 
%How did you do it? State methods.
%What did you learn? State major results. 
%Why does it matter? Point out at least one significant implication.

% Single spacing can be turned on for the abstract
%
{\singlespacing
Computational cognitive models of spatial memory often neglect difficulties posed by the real world, such as sensory noise, uncertainty, and high spatial complexity. However, since cognition and its neural bases have been shaped by the structure and challenges of the physical world, cognitive models should take these into account as well.

This work takes an interdisciplinary approach towards developing a cognitively plausible spatial memory model able to function in real-world environments, despite the sensory noise and high spatial complexity. We investigated how spatially relevant brain areas might maintain an accurate location estimate of mammals, despite accumulating sensory noise, hypothesizing that hippocampal place cells might perform Bayesian cue integration, and that hippocampal reverse replay might play a role in cognitive map correction. We proposed biologically plausible mechanisms facilitating these statistically near-optimal mechanisms, and reported modelling results of single-neuron recordings from rats and behaviour data from humans acquired outside this PhD to support the former, and sketch map accuracy data collected in experiments performed online supporting the latter hypothesis. 

In addition to dealing with sensory noise and uncertainty, in realistic environments, large-scale representations also have to be stored and used efficiently. Hierarchical spatial representations help dealing with large amounts of spatial information by increasing the speed and efficiency of retrieval search and of route planning, as well as facilitating economical storage. It has been suggested that cognitive maps in humans are hierarchical, but the computational principles underlying these hierarchies have received little attention. We investigated features influencing cognitive map structure using collected spatial memory data concerning real-world and virtual reality environments, and proposed a computational mechanism (clustering in psychological space) which might give rise to sub-map structures. We validated our proposed mechanism using spatial memories of human subjects in over a hundred cities world-wide, and implemented a computational model able to predict, in advance, their sub-map structures based on our hypothesis.


{
%\setstretch{1.0}

Based on these insights, we developed a spatial memory module for a general cognitive architecture (the LIDA model of cognition), integrating it with the other cognitive mechanisms built into LIDA. We demonstrated the ability of the resulting model to deal with the challenges of the real world by running it in simulated environments, modelled after our participants’ actual urban environments, using high-fidelity robotic simulation software (including a physics engine) which provides the same interfaces as a real robot. Our LIDA-based spatial memory model could reproduce the spatial representation errors of human participants in different recreated environments, substantiating the plausibility of the computational implementation of our hypotheses.
}

